\chap{Жук} 
\lettrine[lines=3, loversize=0.1]{П}{}ервым, что он~почувствовал, когда очнулся здесь, был запах, пьянящий запах, который полностью захватывал сознание, не~давал чему-то другому занять его место.
Этот запах был самой свежестью.
Он~одновременно был сладок и~горек, была в~нем некая чарующая околдовывающая терпкость и~вместе с~тем притягательная мягкость, были в~нем еще какие-то неуловимые ноты, различить которые было едва возможно, что кружило голову еще сильней.
Этот запах был как весенние бархатные сумерки, в~которые хочется облачить чувствительную кожу, как солнечный летний полдень, который хочется выпить до~дна одним глотком, как бодрящее легким морозцем раннее утро осени, дарующее волшебство своими туманами, как зимняя вьюжная ночь, когда потерял свой ориентир, и~холод и~страх сковывают сердце.
Для Очнувшегося это был запах розового куста после ливня с~грозой, когда природа еще находится в~напряжении после часа молний и~неистовства.
Причем сами цветки куста были не~нежно-розового, а~насыщенного, но~тем не~менее не~приближаясь к~пурпурному, розового цвета.
А~капли, что лежали на~лепестках непременно большие, готовые вот-вот сорваться вниз.
Так для Очнувшегося пахла Любовь.
Та~самая, что редко встречается в~жизни.
В~тот день и~тот миг он~родился, ибо его душа очнулась в~нашем мире после, может быть, тысяч лет пути.
И~первым, что почувствовала душа, был запах истинной любви.
Истощившаяся за~долгие годы она возликовала и~вновь обрела силы.
Потом пробудилось сознание, и~душа отдала ему главенствующее место, ибо так полагалось в~этом мире.
Но~воспоминание об~этом запахе, об~ощущениях, что ~принесла с~собой душа из~другой вселенной, остались, будоража чувства и~память, и~маня...
 
Физически он~был некрасив, велика вероятность, что внутри он~тоже был уродлив для всех жителей своего мира.
Было в~нем что-то не~то, чужеродное, иноземное, не~местное, а~следовательно подозрительное и~всенепременно опасное.
Остальные чурались его, обходили стороной.
Он~не~знал.
Что бывает по-другому.
Недоверие, неведомая злоба окружали его плотной тенью с~самого рождения.
Он~был сам себе противен из-за этого.
Одно он~знал точно: он выродок, монстр, самое низшее и~мерзопакостнейшее из~созданий, потому что оболочка его была уродливой, черной.
Она вызывала приступ тошноты и~паники у~окружающих.
Но~он~держался, а~все потому что обладал странной верой, верой в~то, что ему досталась самая прекрасная в~этом мире душа.
И~нелюбовь всех остальных лишь подкрепляла его веру в~собственную исключительность.
Где-то в~глубине, в~части, которая не~поддавалась сознанию, он~понимал, что душа его не~отсюда и~то, что душа его прекрасна, как и~воспоминания о~том цветке и~запахе, который он~никогда здесь не~чувствовал.
А~еще он~знал, что душа его чиста, как капли росы, которую он~тоже никогда не~видел и~не~ощущал, поэтому мир этот с~такой любовью принял его душу.

Он~жил в~темноте, на~дне, копался в~отходах жизнедеятельности остальных существ.
Душа его страдала с~каждым днем все сильней и~сильней.
Она слабела, теряла силы, стонала, но~не~теряла своей красоты и~чистоты.
И~мир, сжалившись однажды над её~страданиями, решил подарить ей~то~немногое, что имел, ибо он~полюбил душу.
Но~мир этот был сам так уродлив, что не~мог напрямую показать свою любовь.
К~тому же~мир слаб, поэтому он~мог поделиться лишь надеждой.

В~один день тьма чуть отступила.
И~наверху высоко-высоко появилась точка света.
Цветком багряным в~душе расцветала надежда.
Расправив крылья, и~ничуть не~удивившись, что они у~него есть, Очнувшийся, впоследствии взявший имя Жук, полетел.
 
К~душе вернулись силы, она снова стала такой, какой была в~день прибытия.
Он~тоже стал сильным, сильнее, чем когда-либо был до~этого.
И~он~летел ввысь.
В~конце концов он~вылетел за~границы тьмы.
Яркий свет дня больно ударил его по~глазам, но~Жук смог удержаться.
Все вокруг было ослепительно голубым и...
Чистым.
А~наверху, там где-то, бил белый-белый свет.
И~он~полетел высоко-высоко.
Он~летел долго, очень долго, но~так и~не~мог вырваться за~пределы своего мира.
Каждый взмах крыльев давался все труднее и~труднее.
Он~слабел, но~душа оставалась по-прежнему сильной, и~он~продолжал свой полет.
 
Но~однажды в~душу закралась мысль: «~А~вдруг?...».
Доверчивая душа не~знала, что вместе с~этой мыслью к~ней придет и~отчаяние.
Отчаяние убило надежду, но~не~сразу.
Тогда у~надежды был бы~еще шанс вернуться.
Нет, отчаяние убивало надежду медленно, день за~днем.
Минутой за~минутой.
В~один день надежды не~стало, а~вместе с~ней не~осталось больше сил.
А~так как у~жука своих сил уже не~было, то~крылья упали, так больше и~не~поднявшись.
И~жук полетел, только уже вниз быстро, слишком быстро.
 
Удар о~темноту был болезненным, но~он~не~убил Жука.
Он~несколько дней был слепым.
А~душа? А~что же~душа? Отчаяние затенило её~красоту, чуть не~убило чистоту, но~все равно эта душа была самой прекрасной в~том мире.
После неудачи душа начала угасать.
Она не~хотела больше оставаться в~этом мире, но~подняться наверх, к~выходу, она уже не~могла.
С~каждым днем она становилась все меньше и~меньше.
 
И~тут мир, который безумно полюбил эту душу, решился на~то, за~что потом себя так и~не~простил.
Он~решился на~обман.
Он~создал искусственный свет, желтый.
Он~знал, что душа вскоре умрёт, то~есть растворится в~небытие, поэтому и~создал для неё этот прощальный подарок.
Душа увидела свет, и~жук направился к~нему.
Душа была благодарна миру, он~сразу же~раскусила обман, но~от~этого её~благодарность лишь возросла.
Она внезапно поняла, что была любима, хоть и~не~так как хотелось ей.
И~она послала миру свой прощальный дар.
Нежность и~ласку, что помнила в~других мирах, а~единственную ценность, которой владела, аромат розово-розовых роз после грозы.
И~мир был счастлив на~краткое время, а~душа, вернув прежнюю красоту и~чистоту умерла.
\begin{center}
***
\end{center}
Стояла летняя теплая ночь, такая, которая наступает после удушливо-жаркого дня.
Какие-то насекомые все никак не~могли успокоиться, и~их~стрекот разносился по~округе.
На~тёмно-синем небе сверкала круглая луна и~были разбросаны звезды.
Мы~сидели на~веранде, на~старых скамейках, с~ободранными краями.
Я~поджала под себя ноги.
Ю.~допивала кофе с~ромом.
Д.~нервно курил свои особые тонкие ментоловые сигареты.
Над нами испуганно, попеременно грозя перегореть, висела лампочка груша, испуская тусклый желтоватый свет.
Вдруг огромная уродливая тень закрыла свет.
Мы~вздрогнули.
А~Д.~даже выпустил из~рук свою сигарету, и~она упала на~неровный дощатый пол.

 --- Дерьмо!» --- выругался Д.

Это был уродливый навозный жук.
Он~кружил вокруг лампы.
Губы Ю.~презрительно сморщились.
А~мне внезапно стало душно, потому что резкий порыв ветра донес до~нас аромат роз.
И~этот аромат смещался с~дымом сигарет.
На~веранде стало невыносимо находиться.
Мы~пошли по~тропинке к~садовым качелям.
 
~---~То~была похоть.
 
~---~Не понимаю.
 
~---~Сигаретный дым и~розы.
 
~---~Возможно.
 
~---~А~кофе с~ромом~---~страсть.
 
~---~Тоже возможно.
 
~---~А~любовь?
 
~---~А~что любовь?
 
~---~А~как пахнет любовь?
 
~---~Свежескошенной травой.
 
~---~Ромашками.
 
~---~Розами после ливня… 
