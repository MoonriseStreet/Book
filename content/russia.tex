\chap{О снеге в Курске\\
%\begin{center}
\small\textit{и тех мыслях, которые он рождает}}
%\end{center} 
\lettrine[lines=3, loversize=0.1]{Я}{} стою и~вглядываюсь в~небо, ощущая на~своих губах морозное прикосновение снежинок.
И~шепчу нежно нежно: <<Люблю...>>
Произношу это снова и~понимаю, что фраза просто требует продолжения.
Люблю, люблю и...
Ненавижу.
И~сколько бы~ни~произносила <<люблю>>, вмегда за~ним идет <<ненавижу>>.
Что это? А~это просто...
Любовь к~России.
Кажется, словно каждый русский ещё со~школьной скамьи приучается любить свою родину.
Словами Лермонтова.
Вы~спросите, а~разве это любовь? Иностранец, европеец скажет нет.
Русский~---~да.
И~кто прав, и~кто виноват? Да~никто и~ничто.
Но~просто так становится грустно, когда смотришь на~один очередной проект Госдумы, понимая, что твой срок жизни сократился ещё на~лет десять, в~то~время как в~Европе они выбивают свои жизни и~права из~кабинетов министров.
У~нас существует понятие: "Я люблю родину, но~не~государство..." Глупость, скажет иностранец и~будет прав.
Раз ты~любишь родину, ты~любишь её~природу, её~жителей и~их~менталитет, их~аполитичность и~абсентеизм, приводящие к~появлению на~руководящих должностях профанов, не~умеющих даже нормально воровать.
Ты~любишь их, и~их~грязь, мат по~утрам, их~хамство и~чувство собственного превосходства над другими, называемое почему-то патриотизмом.
И~из-за любви своей ты~ничего не~можешь поделать, потому что знаешь и~принимаешь, и~влюблён в~эти недостатки...
Что ж~поделать, повсеместная толерантность? Привила Eвропа нам на~нагу голову бедовую.
Отсюда эта ненависть пополам с~любовью.
Ненависть самих себя за~бездействие.
А~может это просто лень, возведенная в~черту национального характера? Ну~уж~точно не~любовь.
Потому что любовь приносит если не~счастье и~радость, то~облегчение и~в~любом случае любовь заставляет стремиться быть лучше.
А~так, это болезненная, какая-то патологическая привязанность.
Зараза, чума, прищуренным глазом сведшая всех с~ума.
А~наши писатели назвали всеобщее помешательство <<загадочной русской душой>>, чтобы хоть перед ними, правильными, не~боящимися правды, смелыми оправдаться...
И~откуда, но~скажите только откуда, берется это внутреннее убеждение, осознание, что несмотря ни~на~что, у~нас есть душа и~мы~лучше.
Чище, искренней, правдивей.
Чудовищный этноцентризм.
А~может и~правда, и~плевать, что твердит во~все уши глобализация, плевать на~яркие картинки реклам, никто нас не~поймёт.
И~это одиночество нации, целой страны рождает столь губительный, в~первую очередь для нас самих, менталитет...

А~может, это просто фантазии, бред, сказки усталого мозга, размягшего под первым снегом и~серым беспросветным небом.
Почему-то осенью всегда нападает хандра, или аглицкий сплин, которые даже пушкинские стихи и~чашка глинтвейна неспособны излечить.
И~как жить там, где девять месяцев пасмурно на~душе?
Или это просто потому что перед новым годом, все трудности разом отступили, а~ты~ещё не~успел перевести дух и~понять как жить без них.
И~потому что-то мечется в~душе, и~в~столь влажную погоду с~наслаждением ждешь истерику.
И~хочется, чтобы слезы струились по~щекам, и~сердце кровью обливалось.
А~может я~слишком устала, и~сумерки, и~приходящая с~ними тьма утаскивают в~неведомые дали мой, уже слабеющий разум...

Тем не~менее, одна мысль не~дает мне покоя.
Я~хочу, чтобы она прозвучала клятвой и~клятвой навеки осталась.
Я~стану лучше, сильнее, умнее, ради тебя.
Я~буду бороться.
За~себя, за~тебя, за~нас.
Не~потому что ты~или я~достойны этого.
А~потому что я~хочу здоровой чистой любви...

А~вечер однако был поистине замечательным.
Что-то волшебно-морозное медленно и~неотвратимо разливалось в~воздухе, опускалось на~землю вместе с~задумчиво кружащим снегом.
В~такие дни я~всегда поражаюсь метаморфозам, которым подвергается Курск (в~общем-то ничем не~примечательный провинциальный город).
Свежий снег, мягко урывая крыши, стыдливо пряча вездесущую российскую грязь, словно возвращает в~те~годы, когда Курск был губернским, в~романтически воспринимаемое сейчас чувствительными людьми время.
Также меня не~может не~поражать сколь много розоватых домов прячется обычно за~серыми бетонными коробками, построенными в~советский период.
В~туманной дымке снега и~сероватого неба я~и~забываю о~своей ненависти к~этому городу, забываю, как тяжело здесь дышится.
Снег скрывает всё черное и~страшное, все гниюще-опасное, что засело в~Курске с~момента его строительства, так~же, как снег прячет чернозём.
А,~казалось бы,~случилось то~всего ничего...
Снег выпал.
А~я~снова на~пороге того чувства, что некоторые именуют влюбленностью...

А~между тем темп нарастал.
Снег шел уже не~маленькими смешными каплями, а~летел стремительно вниз целыми хлопьями.
И~вот уже не~осталось ничего кроме белого одеяла, укрывшего неоновые вывески, скамьи в~парке, голубей, розовый цвет позорно сбежал уступив белому.
Он~приостановил движение на~улице.
В~парке вообще установилась непроницаемая тишина.
И~в~том задумчивом молчании снег был то~ли~саваном, то~ли~подвенечным нарядом.
Он~то~ли~предрекал скорую погибель маленького городка, то~ли~наоборот обрекал на~светлое булущее и~счастливую и~долгую жизнь...