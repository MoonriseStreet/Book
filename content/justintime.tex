\chap{Вовремя}

\lettrine[lines=3, loversize=0.1]{Е}{}щё в~10~лет случайно прочитав «Анжелику» из~бабушкиной библиотеки (на тот момент все, что могла, я~уже прочла, а~книга в~такой красивой серебристо-голубой обложке с~таким красивым тиснением, так и~манила прикоснуться к~ней, снять с~полки, раскрыть страницы и~погрузиться в~ее~пока ещё мне неведомый мир), я~поняла, что существуют вещи: книги, слова, фильмы, эмоции -~для которых есть свое правильное время.
Это очень важно, чтобы о~некоторых понятиях мы~узнавали в~нужное время, когда наш эмоциональный и~умственный опыт позволяет нашему же~сознанию все переварить, правильно понять и~усвоить без надрывов, потрясений и~получения детских травм.
Говоря про Анжелику, я~получила, на~мой собственный непрофессиональный взгляд, детскую травму.
Я~ещё очень долго не~могла читать ничего, содержащего хоть малейший намек на~секс, и~к~тому же, выросла жуткой ханжой.
По~прошествии времени, повзрослев, я~поняла ещё один важный момент касаемо понятия «вовремя»: быть уместным, пунктуальным, даже просить о~помощи у~какого-либо человека в~удобное и~приятное для него время~---~важный и~полезный навык, который стоит развивать всем без исключения.
А~потому на~суд читателей я~предлагаю свою абсолютно правдивую историю, а~какие делать из~нее выводы и~делать ли~их~вообще, решать вам.
 

Сразу следует сказать, что отношения с~матерью у~меня нормальные, ну… как у~всех детей, чьи родители, страдают гиперопекой.
Даже когда мне стукнуло восемнадцать, в~мамином присутствии я~не~переставала чувствовать себя маленькой и~глупой девочкой, которая ничего не~знает о~злом и~страшном реальном мире.
И~вот так случайно вышло что, когда я~уже училась в~университете, нам с~мамой пришлось жить вдвоем, считай, запертыми в~трехкомнатной квартире.
Конечно, за~то~время, что я~жила от~нее отдельно, ее~акции на~рынке моих авторитетов сильно упали.
Где-то глубоко внутри, на~интуитивном уровне, она почувствовала это и,~как всякий родитель бессознательно захотела вернуть им~прежние позиции.
Вспомните своих бабушек и~дедушек и~то, как они постоянно, капризно требуют вашего времени и~заботы.
Мама начала делать также: она неотступно и~неусыпно гонялась за~моим вниманием.
Она вечно в~чем-то нуждалась, ей~всегда была нужна моя помощь: будь то, простое открытие банок с~солениями или же~желание посидеть вместе перед телевизором и~скрасить вечер тяжело трудившегося весь день человека.
И~сколь много внимания я~ей~ни~уделяла, как много бы~времени ни~проводила вместе с~ней, ей~всегда было мало.
В~конце концов, у~меня была и~своя жизнь, и~свои дела.
И~в~конце-то концов мне нужно было делать домашку по~универу! Проблема была ещё и~в~другом: она всегда просила помощи именно в~тот самый момент, когда я~была очень занята.
Пример? Да~легко, пожалуйста.
Ей~нужно было распечатать документы вот кровь из~носу прямо сейчас, которые могли подождать и~до~вечера, именно в~тот момент, когда я~решала домашку по~дифференциальным уравнениям и~которая у~меня не~сходилась, а~дедлайн был через пару часов.
Да, я~все сделала, да, я~распечатала ей~эти документы, но~с~каким скандалом и~с~какой нервотрепкой.
 

И~так продолжалось довольно долго, отгремел май со~своими грозами, в~июне чуть распогодилось, солнце робко проглядывало сквозь тучи и~радовало взгляд.
Наступила пора, когда наконец-то можно было гулять, да~и~режим самоизоляции, то~бишь карантин, перестал быть таким строгим: открывались торговые центры, магазины, повеселевшие люди всё большими и~большими толпами высыпали на~улицы.
Наступала удивительная пора… Для всех, кроме студентов.
Для нас, так как я~все ещё отношу себя к~этой братии, июнь месяц означал одно~---~неумолимое приближение сессии, а~перед ней -~зачетную неделю.
Я~благополучно зазубривала для первого экзамена по~английскому пять огромных текстов на~1,5 тысячи слов и~около двадцати штук маленьких.
Успешно отговариваясь подготовкой к~экзаменам от~выполнения странных и~глупых просьб и~тем не~менее помогая с~адекватными делами, я~потихоньку готовилась к~сессии.
В~середине недели ударила жара: плавился асфальт, плавились люди, которые по~нему ходили, все уже забыли о~том, как недавно жаловались на~дожди и~холод, и~втайне мечтали о~том, чтобы жара кончилась.
К~концу недели, то~есть в~пятницу сидеть дома (даже под кондиционером!) и~в~сотый раз повторять: «~…~who also learned his trade on~the streets…” ,~стало совсем уж~невыносимо.
И~вечерочком, часиков в~семь, я~выбралась погулять.
Я~шла по~улицам в~легком летнем платье, чувствуя себя Маленой из~одноименного фильма, наслаждалась застывшим от~жары розово-фиолетовом небом и~попутно проговаривала про себя фразы из~английских текстов, хотя и~в~моей голове, и~в~моем сердце звучали совершенно другие слова.
 

Но~вдруг красоту вечера перечеркнул противный пиликающий звук телефона.
Звонила мама.
Я~специально перевела телефона из~беззвучного режима в~нормальный.
Интуитивно предполагая, что могу ей~снова понадобиться.
Я~взяла трубку.
 

--- Да, мам.
 

--- Ты~дома? Ты~мне будешь нужна минут через пять.
Это срочно.
Слышишь? Срочно!


Я~ускорила шаг и~параллельно пыталась разузнать, зачем я~ей~понадобилась.
Стоит упомянуть, что от~дома я~отошла хоть и~недалеко, но~ощутимо (Я~находилась на~расстоянии пятнадцати минут ходьбы).
Оказалось, что она тащила откуда-то ходунки для дедушки, и~я~ей~была нужна, чтобы вынести из~дома ключи от~машины, ведь она была в~полной уверенности, что весь вечер я~просидела дома, как прилежная девочка, зубря тексты.
Пока я~пыталась ей~объяснить, что я~не~дома и~почему и~как вообще так вышло, что я~появилась вдруг где-то вне дома, я~перешла на~бег.
Потом я~слушала, как мне выговаривают, что вечно я~не~рядом, когда нужна, что помощи от~меня днем с~огнем не~сыщешь, я~начала бежать ещё быстрее.
Я~чувствовала, что мне кровь из~носу надо быть дома через пять минут, чтобы сохранить статус-кво наших отношений.
Я~сбросила вызов и~побежала изо всех сил, но~очень скоро выдохлась.
А~мне оставалось ещё полпути.
И~тогда в~мою бедовую голову пришла, как мне тогда казалось, гениальная идея.
Я~видела, что к~остановке на~другой стороне улице подходит троллейбус, идущий в~мою сторону.
И~в~тот момент меня не~волновало, что налички у~меня не~было, что в~карманах не~нашлось бы~даже одной самой завалявшейся пятидесятикопеечной монетки.
Светофор поменял свой цвет на~«зеленый», все складывалось удачно, я~вбежала на~пешеходный переход, не~посмотрев как следует по~сторонам.
Вот и~все.
Конец.
Встречный водитель не~имел привычки тормозить перед остановками и~пешеходными переходами, я~выскочила внезапно, он~этого не~ожидал.
Меня сбило на~скорости 80~км/ч.
Как легко догадаться, насмерть.


\vspace{1cm}
P.~S.~Она узнала о~смерти дочери только через полчаса.
Когда дочь не~появилась около подъезда ни~через пять минут, ни~через пятнадцать, Она забеспокоилась.
Она звонила и~звонила, пока один из~гаишников не~взял телефон из~маленькой черной бархатной сумочки с~золотым бисерным узором посередине.
Он~ответил, вот только к~ответу Она не~оказалась готова.

