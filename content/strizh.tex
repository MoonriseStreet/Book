\chap{***}

\lettrine[lines=3, loversize=0.1]{Э}{}то знакомое многим, я~уверена, чувство, было для меня новым.
Я~до~сих пор не~решаюсь дать ему точное название.
Ведь как-то конректно назвать его означает для меня потерять навсегда его необычность и~своеобразную магию.
Я~также не~могу его никак охарактеризовать.
Но~я~всегда узнаю его,когда оно приходит.
И~мысленно легкой улыбкой приветствую его.
 
Это случилось… 
Да, был солнечный день.
Лето.
Почему-то я~всегда была уверена, что для таких событий больше подходит середина осени с~её~утренними туманами и~моросящими дождями.
Это казалось мне, не~знаю, более правильным, книжным.
Как в~романах.
Но~все было наоборот.
Замерший душный воздух летнего вечера, розовато-фиолетовая полоска будущего заката.
Зелёный шум старых больших каштанов и~суетливые веселые крики стрижей, взмывающих в~синее небо с~красно-кирпичных крыш.
Мы~лениво лежали на~диване, уставшие от~дневной жары.
Балконная дверь была закрыта, позволяя кондиционеру охладить квартиру.
Надоедливый шум телевизора смешивался с~дымом сигарет и~кисловатым запахом старого паркета.
Все это создавало поистине удушливо ленную атмосферу.

Из~полудрёмы меня вывел истеричный визг и~хлопанье крыльев.
Неторопливо вышла к~источнику шума.
Он~находился на~балконе.
Как ни~странно, но~мама, находившаяся до~этого на~кухне, тоже услышала визг и~пошла на~балкон.
А~там на~деревянном подоконнике, отчаянно лупя своими красивыми черноватыми крыльями, рвался на~свободу стриж.
Когда наблюдаешь за~стрижами с~девятого этажа, они кажутся легкими черными точками.
Но~тут передо мной билось живое существо.
Его черные глаза бусины смотрели только на~небо, отгороженное от~него стеной стекла.
Блики в~глазах бешено метались, крылья не~останавливались ни~на~секунду.
В~первое мгновение я~опешила, залюбовавшись черными перьями, гладко подогнанными друг к~другу.
Но~потом страх волной от~стрижа перешел ко~мне.
Его страх.
Страх оказаться вдали от~синей бездны, от~потоков ветра, всегда струящихся под крыльями.
Он~боялся нас, людей,странных существ, которые не~летают.
Уверена, его голова кружилась от~совершенно незнакомых и~непривычных запахов нашего балкона.
Сердце его стучало быстро, очень быстро, заставляя работать крылья постоянно.
Да~и~сердцебиение птиц намного быстрее человеческого.

В~своей панической боязни стриж, по~моему, ничего не~видел и~шарахался от~всего.
Когда я~подошла ближе он~взмахнул крыльями, попытался оторваться от~полированной поверхности подоконника и~упал на~пол.
Но~он~не~перестал бешено бить крыльями.
Я~быстро нагнулась, взяла его в~руки.
Я~чувствовала его крылья, перья, его бьющееся сердечко.
Его отчаяние.
Его надежду.
Он~был очень легким и~таким хрупким.
Казалось, его легко сломать.
Но~одно я~также поняла в~этот миг: сломать просто его не~получится.
Он~бил своими крыльями меня по~руками, не~останавливаясь.
Он~отвергал мои руки, хотя они были единственным, что могло ему помочь.
Он~сопротивлялся помощи.
Неистово рвался к~свободе, которую по~его мнению у~него хотели отобрать.
За~это время мое сердце стало биться с~такой же~скоростью как и~сердце стрижа.
Мне казалось, что мы~разделяли одни и~те~же~чувства.
Тот же~страх и~то~же~отчаяние.
Буквально через секунду я~высунула руки из~окна, резко встряхнула их~тем самым давая стрижу возможность оттолкнуться.
Он~взлетел и~немного криво полетел ввысь.
Я~опустилась на~пол, тяжело дыша.
Страх и~отчаяние ушли.
Я~боялась за~стрижа.
Я~боялась вместе с~ним.
Все это заняло не~более десяти секунд, но~для меня это затянулось на~несколько часов.
Теплый ветерок игриво шевельнул курчавую прядь, а~я~подумала: <<Лети, птичка>>...

