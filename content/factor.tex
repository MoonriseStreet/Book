\chap{Человеческий фактор}


Мне было пять лет, когда я~осознала простой факт: <<Мир жесток и~несправедлив>>.
Жить дальше, принимать решения, заводить друзей и~прочее я~начала исходя из~этого посыла.
Каждый сам за~себя.
Люди эгоистичны по~своей натуре, и~в~этом нет ничего плохого.
Да~и~вообще понятия добра и~зла придумало само человечество из~страха перед жизнью и~своими истинными желаниями.
Есть только ты~сам, твои нужды и~стремления, а~все остальные стремятся всеми способами, всеми правдами~и~неправдами помешать тебе получить то, что ты~так жаждешь.
И~нет, я~не~росла в~неблагопулочной семье в~криминальное районе какого-то провинциального города в~центре Сибири.
Моё детство и~отрочество прошли вполне благополучно, и~в~достатке, и~радости.
Может, просто пересмотрела с~бабушкой <<Пусть ГОВОРЯТ>> в~детстве и~привыкла к~тому, что реальный мир~---~грязная штука, где всех и~вся волнует только личная выгода.
В~общем и~целом, наверное, так и~есть.
Да~вот только в~это можно поверить, если верить в~статистику, если смотреть новости по~первому каналу, если слушать бабулек на~скамейках у подъезда, но... Но~статистика такая штука, которая мешает взглянуть на~вещи конкретно, не~обобщённо, не~замыливая взгляд.
Статистика ошибается.
Когда дело касается конкретных людей, конкретных ситуаций, когда проблемы решаются на~уровне людей, а~не~государств, статистика имеет свойство ошибаться на~все сто процентов.
 
\vspace{7mm}

Егор возвращался домой.
Ну~как домой, он ехал обратно на~съёмную и~не~очень уютную, но~тем не~менее свою квартиру.
И~было в~этом что-то радостное и~придающее уверенности в~себе.
Вот уже год, как он~съехал от~родителей, и~это событие до~сих пор грело ему душу.
Кто-то, усмехнувшись, может сказать: <<Подумаешь, от~родителей переехал в~двадцать лет, вот уж~достижение, так достижение!>> Но~для Егора, чьи родители страдали гиперопекой, это была самая настоящая победа.
Он~уговаривал их~на~раздельное проживание в~течение трех лет и~это несмотря на~то, что уже как два года он~сам обеспечивал себя и~свои маленькие прихоти.
Многие его так называемые <<друзья>> были уверены в~том, что Егор до~сих пор сидит на~шее у~своих родителей, и~он~не~спешил их~разочаровывать в~этом.
Он~подработатывал в~отцовской фирме, немного занимался фрилансом и~на~эти достаточно неплохие деньги вполне сносно существовал.


Егор как раз и~ехал домой от~одного из~этих <<друзей>>, тот праздновал свой день рождения, на~котором Егор из~года в~год должен был присутствовать с~самого детства, по~причинам ему неведомым.
Но~отец требовал поддерживать дружбу с~младшим сыном мэра, и~Егору приходилось скрепя сердце ехать <<веселиться>> к~людям, которых он~не~понимал и~более того не~желал понимать.
Ему были далеки те~мысли, которые они высказывали, те~развлечения, которые приносили им~удовольствие.
К~двадцати годам он~уже достаточно насытился жизнью, чтобы безудержные безбашенные попойки до~6~утра, посон которых не~остаётся ничего, кроме невыносимой головной боли, продолжили приносить ему радость.
Да~и~от~бесконечной вереницы девиц, сменяемых со~скоростью перчаток, чьи имена он~устал запоминать, постоянно болела голова.
Поэтому в~конце концов он~просто как-то написал девушке, влюблённой в~него по~уши аж~с~их~седьмого класса, и~они начали встречаться.
 

Это было крайне удобно: не~нужно было каждый раз притворяться очарованным красотой неизвестной дамы, приглашать на~свидание, а~затем соблазнять, чтобы в~итоге получить секс.
Всё было рядом, все было под рукой, а~бонусом шла её~безграничная любовь, нежность, забота, ласка и~понимание.
А~от~него самого практически вообще ничего не~требовалось: не~изменять, не~грубить, не~бить.
И-де-а-ль-но.
Егор уверял сам себя в~том, что просто исполняет наивную девичью мечту и~помогает Оле закрыть гештальт на~свой счёт, где-то в~глубине души понимая, что поступает неправильно.
Нет, не~нечестно, он~был убеждён, что их~с~Олей отношения~---~взаимовыгодная сделка, пусть и~без перспектив, от~которой каждый получает то, что ему нужно.
И~всё же~каждый раз, когда Оля открывала ему дверь, и~глаза её~сияли, когда он~поутрам брился, а~она наблюдала за~ним и~на~его вскинутые брови и~вопрос~<<Что?>> улыбалась уголками губ, говорила: <<Ничего...>>, и~вздыхала тихо и~радостно, в~душе его скребли кошки, потому что он~не~мог разделить её~чувства.
Он~её~банально не~любил, но~держал рядом, мешая ей~найти свое настоящее счастье, с~человеком, который будет любить её~также сильно, как и~она его.
 

Погрузившись в~свои мысли, Егор не~заметил, как на~смену сумеркам пришла непроглядная ночная мгла.
Она подступала медленно, но~накрыла неожиданно, не~оставив и~шанса сбежать от~её~мягких объятий.
Чертыхнувшись, и~обругав губернатора на~чем свет стоит за~то, что федеральная трасса, по~которой Егор ехал из~снятого в~области особнячка в~К., этой весной осталась без фонарей, то~есть полностью без света.
Ещё раз выместив душу, на~новом губернаторе, которого им~рекламировали, как прекрасного дорожника, Егор врубил дальний свет фар и~сбросил скорость до~100~км/ч.
Не~то, что~бы он~совмевался в~своих водительских способностях и~умении быстро соображать в~экстремальных ситуациях, но~всё же~на~дороге ночью может случиться что угодно, а~так безопаснее, хоть эту конкретную часть трассы, Егор знал как свои пять пальцев, гоняя здесь на~отцовском джипе с~11~лет.
 

Новенький матово-черный порше, подаренный дедушкой на~двадцатилетие, приятно урчал под капотом, по~радио звучало что-то попсово-расслабляющее, за окнами неуловимо быстро мелькали страшновато-угрюмые из-за голых ветвей силуеты деревьев.
Егор позволил невесёлым думам потихоньку отпустить себя и~немного расслабился.
Его мысли плавно перескочили на~порше, он~невольно усмехнулся.
Столько из-за этой машины было скандалов и~споров.
Мать Егора кричала на~дедушку, своего отца, что тот слишком сильно балует внука, что из-за его разлагающего влияние, Егор вырастет избалованным (ага, в~двадцать лет-то), не~знающим цену деньгам.
Но~тогда дедушка сказал ей, что автомобили этой марки, признаны самыми надёжными в~мире и~что даже если он~попадёт в~аварию на~этой машине, вероятность получить тяжёлые травмы будет гораздо ниже, чем в~заработанной на~собственные деньги девятке или того хуже Ладе Калине.
 

Внезапно на~горизонте Егор уловил ярко-жёлтую точку, которая постепенно увеличивалась в~диаметре, пока не~превратилась в~две жёлтые опасно мерцающие точки.
Егор внутренне напрягся и~ещё больше сбросил скорость.
Вскоре до~Егора дошло, что машина, ехавшая ему навстречу, буквально летит по~Егоровой полосе, выжимая все 150-160.
В~первые секунды, осознав это, Егор растерялся, он~не~знал, что ему делать.
А~потом... А~потом он~понял ещё кое что, что заставило его буквально похолодеть от~страха.
Огни машины располагались низко, слишком низко.
Такое могло быть только у~табакерки на~колёсиках, под~названием Ока.
Егор успел даже рассмеяться и~удивиться тому, как что-то настолько древнее и~дышащее на~ладан способно разгоняться до~ста пятидесяти.
А~потом страх снова накрыл его.
Лихорадочно соображая, что же~делать, Егор тем не~менее понимал один простой факт: если позволить Оке врезаться в~него, то~во-первых, он~ни~в~чем не~будет виноват, не~он~гнал по~встречке, спеша на~свидание со~смертью, а~во-вторых, водителю и~тому, кто сидит на~пассажирском сидении, не~жить.
Егор пострадает не~сильно, порше лишь поцарапается, ну~может, бампер помнется, а~вот Ока при столкновении сомнется, и~водитель, и~пассажир умрут в~тот же~миг.
В~начале Егор думал, что позволит Оке врезаться в~себя, пусть и~чувствовал, что поступает неправильно, как в~ситуации с~Олей.
Но~в~самых последний миг, когда их~разделяли какие-то метров десять, будто повинуясь незримой силе, резко повернул руль налево, сам вылетя на~встречку, и~на~пару секунд потерял управление автомобилем, пока облегчённо переводил дух, осознавая, что в~все остались живы.
Этих минут хватило, чтобы машину Егора занесло на~обочину.
Никто не~ожидал, никто не~знал, что один местный предприниматель выкупил эту землю около трассы для строительства мотеля, и~обнес свою территорию бетонным забором.
Егор и~его порше на~полной скорости врезались в~этот забор.
 
\vspace{7mm}

Егор вышел из~комы на~пятые сутки.
Водитель Оки, пожилой слесарь-пьяница и~его жена остались живы и~даже не~пострадали.
Они благодарны провидению за~то, что смогли этой ночью вернуться к~детям.
Егор получил обширные внутренние повреждения, ушиб почти всех внутренних органов, в~том числе ушиб лёгких и~сердца, разрыв селезенки и~нескольких сосудов.
Восстановление займёт пару лет, о~полном выздоровлении говорить пока рано.
Сейчас Егор вернулся в~отчий дом под присмотр любящих родителей и~компетентных врачей.

\vspace{7mm}

\textit{Основано на~реальных событиях.\\ Имена героев и~место действия изменены.}
