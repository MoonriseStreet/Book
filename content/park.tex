\chap{Парк}
 
\lettrine[lines=3, loversize=0.1]{Э}{}то был летний вечер субботы.
Хмурые дневные тучи разбежались, и~робкие солнечные лучи просвечивали сквозь листья деревьев.
Из-за этого свет в~парке был зеленым.
Как и~обычно играл оркестр.
Музыканты принесли с~собой стулья и~расположились в~центре парка, играя музыку военных лет.
Откуда-то появились бабушки и~дедушки в~нарядных костюмах.
Многие дамы могли похвастаться причудливыми шляпками.
Площадка в~середине парка вскоре была заполнена танцующими парами, только некоторые из~которых попадали в~ритм вальса.
Они двигались медленно, получая удовольствие и~неспеша разговаривая.
На~детской площадке напротив было шумно и~весело.
Дети кричали, смеялись.
Плакали.
Слышались удары ног по~мячу, скрип старых качелей.
От~центральной части, словно радиусы окружности отходили дорожки.
По~этим дорожкам прогуливались люди средних лет, за~которыми семенили маленькие собачки, ростом меньше среднего кота.
Их~огромные глаза были пусты, а~носы водили повсюду, пасть не~закрывалась, издавая громкий лай.
Обгоняя собак, мчались велосипедисты в~разноцветных шлемах.
Им~хотелось побыстрее пересечь парк полный неразумных детей и~их~родителей.
Но~часть дорожек была и~вовсе пуста.
Одну из~них Она и~выбрала именно из-за ее~пустоты.
Здесь все было также.
Огромные клены и~каштаны своими не~менее огромными листьями закрывали небо и~не~пропускали лучи.
Чуть съехав с~дорожки, выложенной плиткой, можно было попасть на~дорожку, протоптанную людьми.
Недалеко от~этой дорожки находился большой старый пень.
Местами он~был покрыт лишайником и~мхом, пара длинных трещин рассекала его слева, но~он~стоял на~этом месте уже много лет и~многие люди садились на~него, чтобы передохнуть.
Девушка повела свой велосипед к~этому пню, и,~поставив его так, чтобы он~не~свалился, уселась на~пень.
Нервными пальцами она открыла рюкзак, что до~этого висел у~нее на~спине, и~достала черный ком проводов, в~который превратились наушники.
Она начала теребить их, но~в~итоге запутала еще сильнее.
Тогда она достала телефон, но~тот выскользнул из~дрожащих рук и~упал в~траву.
Руки опустились на~колени.
Она подняла голову и~посмотрела на~небо, которого не~было видно.
Из~глаз покатились слезы.
Они размазали тушь.
Ей~было больно, очень больно в~душе.
Там, внутри неё все смешалось.
В~голове обрывками висели мысли.
Верить или не~верить? Доверять или нет? И~почему так больно? Почему хоть раз не~может пройти все нормально? Привязываться нельзя.
Потом слишком неожиданно.
Забываешь, что один.
Но~иллюзии рушатся...
И~от~этого еще больней.
Почему так? Можно ли~доверять человеку, уже совравшему? Врать нужды не~было.
И~врет ли~сейчас? А~может стоит вспомнить старую добрую тактику и~забыть? Со~временем я~все узнаю.
Может, не~стоит так глубоко копать? Мои эмоциональные горки меня уже достали.
Или я~просто нашла еще одну причину, чтобы ненавидеть себя? 

Внезапно слуха коснулся лай.
Она вернула голову в~нормальное положение и~посмотрела вперед.
Там три дерева стояли полукругом, создавая что-то наподобие сцены.
И~на~этом месте мелкая собачка, чья голова еле виднелась из~травы, гавкала на~свою хозяйку, женщину лет сорока с~короткой стрижкой и~в~спортивном костюме.
А~солнечные лучи, которым все же~удалось пробиться сквозь листья и~ветки, образовывали золотые лужи на~зеленой траве.
Эта картина заставила девушку улыбнуться.
Она почувствовала.
Что находится в~самом центре покоя.
Этот парк, это место в~нем были такими мирными.
Окружающий мир был тих и~доволен.
Но~внутри, что творилось внутри… Там бушевал пожар, сердце требовало испытаний, страданий, боли, преград, что нужно сломить.
Молодое тело хотело лишений и~нужды.
Все ее~существо жаждало проверить себя, испытать.
Но~как назло вокруг ничего не~происходило.
Она была в~самом сердце мира.
И~так было всегда.
Она жила среди людей, уже хлебнувших горя.
Им~не~хотелось ничего, кроме покоя.
Они радовались каждому дню.
И~они мечтали, чтобы и~она и~все ее~поколение никогда не~испытывали чего-то подобного.
Но~молодое сердце и~молодая кровь жаждали войны.
Она не~хотела этой войны в~сердце, война родилась невольно, без желания.
Молодым нужны приключения, опасность, всплеск адреналина.
Она хотела того же, что и~окружавшие её: обрести покой в~душе.
Но~видела она только один выход: сейчас вокруг нее покой, а~внутри война~---~значит, войну нужно вывести наружу.
Она пока не~знала как, но, осознав, твердо уверилась в~том, что найдет способ.

~
Она опустила голову и~позволила звукам из~внешнего мира войти в~нее.
Она попыталась охватить все, что ее~окружает.
Сначала она впитала собственное сердцебиение, потом стрекот насекомых в~траве, лай собаки.
Она закрыла глаза и~прислушалась.
Смех детей на~площадке, крики птиц, разговоры прогуливающихся… Она была в~центре мира, но~сражение бушевало в~ней…
 
А~оркестр играл прощание славянки.
Пары медленно кружились.
Подул легкий ветер, который приподнял пару шляп.
Птицы и~велосипедисты стремительно неслись вперед.
Это был обычный летний вечер… 
