\chap{Труд
} 
Стоял конец ноября.
Кое-где уже лежал белыми пятнами первый снег.
Жуткий пробирающий до~костей ветер гнал случайных прохожих по~домам.
Холодное, неуютное время.
Иногда холод улиц становится настолько сильным, что люди промерзают внутри, холод поселяется в~их~сердцах, делая равнодушными или это только оправдание? 
~
Той осенью им~долго не~могли включить отопление.
Феня сидела в~сером пледе на~кухне рядом с~обогревателем, от~которого пахло старостью и~прошлым веком, и~уныло смотрела в~окно.
Серые сумерки, в~том году сумерки были только серого, сгущались за~окном, но~Феня все равно успела разглядеть две одинокие снежинки медленно кружащиеся каждая в~своем танце.
Может, вместе им~было бы~теплее?~---~кольнула ее~мысль.
И~она стала ждать, когда же~снежинки додумаются, что им~вместе лучше, но~нет, они кружили каждая по-своему и~не~хотели сближаться.
Каждая хотела, чтобы другая сама делала первые шаги навстречу не~потому что у~них были гордость или потому что их~разъединяли обиды прошлых лет.
Нет, каждой из~них было попросту лень сделать первый шаг, и~каждая ждала, что у~другой этой лени поубавится.
Феня хотела бы~встать, как-то подтолкнуть их~друг к~другу, но… Но… Но…
 
Тихо в~двери прозвучали ключи.
Кто-то вошел в~квартиру и~не~снимая с~себя ботинок и~теплое пальто прошел на~кухню.
Дима посмотрел на~Феню, заснувшую прямо на~неудобной кушетке и,~только хмыкнув, подошел к~окну.
Там уже было темно и~только на~небе светились звезды.
И~как они не~устают!~---~восхищенно подумал он, продолжая смотреть на~небо.
Он~бы~точно уже устал.
Он~не~был таким сильным, как звезды и~не~был бы~таким же~до~конца преданным работе, как их~свет, который мы~видим.
Ведь сами звезды могли уже умереть, а~свет их~все идет и~идет.
В~душе Дима всегда завидовал силе звезд.
Отойдя от~окна, он~посмотрел на~стол, где Феня могла бы~оставить для него термос с~горячим чаем…
 
Феня проснулась с~утра и~поняла, что Дима уже ушел, а~она всю проспала на~этой ужасной кушетке и~теперь у~нее нестерпимо болела спина.
Вспыхнула мысль, что Дима мог и~отнести ее~в~спальню, но… Она вспомнила вчерашние снежинки и~поняла, на~что похожи их~с~Димой отношения.
Они не~предпринимают абсолютно никаких действий, потому что зачем ~нужно перетруждаться.
Потому так и~холодно в~их~квартире.
Всегда было холодно.
Она сколько себя помнила, никогда особо не~хотела перетруждаться, утомлять себя.
Собственно поэтому она и~выбрала Диму: он~был такой же.
Нет, в~самом начале она кое-что предпринимала, но~не~получив отклика быстро бросила это дело.
Хотя в~душе знала, что с~первого раза ничего не~получится и~нужно пробовать еще и~еще.
Делать, думать.
Искать способ решения.
Но~зачем? Это потребовало бы~ее~сил.
А~сейчас сил итак нет.
Куда они улетучились, она не~представляет.
Внезапно ей~стало страшно.
В~этом мире нужно делать, иначе данные тебе силы заберут.
А~она не~хотела делать, никогда не~хотела.
Ей~не~нужен такой мир, мир, где ее~хрупкое стрекозье существо заставляют действовать.
И~она с~определенной долей хладнокровия приняла решение.
Быстро сунув ноги в~ботинки, она вышла в~осень, забыв закрыть дверь.

~

~

~
Дима вернулся домой, когда уже стемнело, а~темнело рано, и~учитывая, что добирался он~по~пробкам, то~темнота успела прибрать город к~своим рукам.
В~квартире не~горел свет, на~открытую дверь внимания он~не~обратил и,~не~разуваясь, сразу пошел к~спальне, решив, что Феня опять спит на~кухне.
Дня через три он~все-таки понял, что жены нет дома и~начал ее~искать.
Родных у~Фени не~было: мать умерла год назад, и~кроме него она по~сути была больше никому и~не~нужна.
Единственное что он~сделал -~это обратился в~полицию и~привлек добровольцев, а~потом заперся у~себя в~квартире.
Многие решили, что это от~большого горя, но~он~не~стал их~разочаровывать.
Ему было все равно.
В~их~семье давно царила пресловутая нелюбовь.
Он~просто использовал это как повод не~ходить на~работу и~спокойно лежать на~диване.
Отопление к~тому времени уже успели дать.
Однажды в~квартиру все-таки позвонили.
Он~представился как Женя.
Женя был высоким, слишком худым, постоянно горбился.
Куртка была ему широка в~плечах и~смотрелась несуразно.
Голос был осипший, но~слишком писклявый для его возраста.
Женя был волонтером.
И~он~попросил Диму ехать с~ним.
В~тот день волонтеры осматривали какой-то парк.
Дима так ничего и~не~понял из~Жениных объяснений.
В~прочем ему было все равно.
И~только там, в~парке, Дима смог понять, какой человек этот Женя.
Он~был огромным.
Он~превышал свой рост и~свое тело и~голос его не~казался писклявым, а~самым правильным, и~куртка не~выглядела больше несуразной.
Четко раздавал он~команды, координировал действия других.
Грамотно руководил ими, ни~одного лишнего слова, ни~одного непрофессионального движения.
В~его подчинении находилось столько людей, и~они уважали его, уважали его слова, его опыт.
Дима удивленно смотрел на~него и~восхищался им.
Этот человек был занят делом и~в~своем деле он~был велик.
Остальные волонтеры тоже хоть и~уступали Жене, но~выглядели настоящими великанами по~сравнению с~Димой.
Внезапно он~ощутил себя пятилетним ребенком, слишком маленьким, попавшим на~работу к~взрослым.
На~морозе лица этих людей раскраснелись, появилось особое выражение~---~выражение веры в~свое дело.
И~лица их~стали прекрасны.
Дима сказал, что ему вдруг стало нехорошо, и~медленно побрел к~Жениной машине.
Его провожали такие сочувствующие и~такие прекрасные взгляды.
Сев в~машину, он~посмотрел в~зеркало заднего вида и~отвернулся.
До~того безобразным было у~него лицо.
..
 
Дмитрий Иванович вставал не~по~будильнику, но~всегда вовремя.
За~столько лет жизни в~одиночестве, когда он~сам себе хозяин, он~успевал прекрасно выспаться.
На~кухне он~опять включил телевизор и,~запивая бутерброды кофе, любовался ведущей новостей.
Она была так воодушевлена и~была такой молодой, что он~одобрительно улыбался.
Потом он~посмотрел на~кушетку и~вспомнил о~Фене.
На~самом деле.
Неизвестно так ли~все было на~самом деле, как помнил это Дмитрий Иванович.
Скорее он~придумал ту~Феню и~тот ноябрь, посмотрев еще в~мае фильм Нелюбовь.
Кино он~любил, потому что для его просмотра не~нужно было ничего кроме дивана и~времени, а~времени у~Дмитрия Ивановича было всегда много.
Это был фильм Алексея Звягинцева.
Обычно он~такое не~смотрит, но~в~тот раз почему-то решил.
И~вряд ли~бы~он~понял, о~чем этот фильм, если бы~не~уход Фени тогда, десять лет назад.
А~посмотрев Нелюбовь, он~понял, одну очень важную вещь.
Счастье в~отношениях нельзя получить просто так.
Даже если вы~безумно влюблены, даже если вы~дышите только друг другом счастье нужно заработать.
И~чтобы в~паре было обещанное одним счастье, двоим нужно трудиться.
Он~понял, что даже в~своей семье нужно работать.
А~с~Феней они просто сосуществовали в~одной квартире.
 
На~работу Дмитрий Иванович теперь шел пешком.
Это оказалось быстрее, чем пользоваться общественным транспортом.
Полчаса спокойного шага и~он~был на~работе.
Был он~кем-то вроде менеджера по~продажам, и~держали его там из~жалости, давая заполнять какие-то бумаги время от~времени.
А~в~основном он~сидел на~стуле перед компьютером с~девяти до~пяти с~часовым перерывом на~обед.
Дмитрий Иванович любил туман и~когда он~шел на~работу туман был его спутником.
Он~окутывал с~ног до~головы маленькую фигурку, делая ее~совершенно незаметной.
Прохожие и~так бы~не~замечали его, спеша по~своим делам, волнуясь из-за предстоящего дня, не~подозревая, что их~ждет завтра.
Дмитрий Иванович словно оказывался тогда в~другом мире, параллельной вселенной, ведь он~точно знал, что ждет его завтра.
Его жизнь имела свой ритм, она никогда не~менялась, была до~омерзения постоянной.
И~Дмитрий Иванович становился постепенно таким же, как и~его жизнь.
У~него не~было цели, не~было работы, занятия.
И~я~говорю даже не~о~работе, которая позволяет нам добывать средства к~существованию, а~о~деле, о~том, чтобы его увлекало, о~чем бы~ему хотелось говорить часами, о~том, что вызывало бы~блеск в~его глазах.
Но~ничего такого в~его жизни не~было никогда.

~
Придя на~работу, он~скинул пальто, которое так и~не~менял с~момента ухода Фени.
Это было уже очень старое пальто, местами выцветшее, кое-где в~катышках, несуразно смотрящееся на~его фигуре.
Фигура его к~тому времени сильно изменилась.
Из~вполне приятного молодого человека среднего роста с~нормальным телосложением в~сорок с~чем-то лет он~превратился в~старичка с~худыми неприятными руками и~маленьким брюшком.
Его светлые поредевшие волосы неопрятно покрывали голову.
Он~постоянно потирал руки, словно муха, что служило вечным раздражителем его начальника.
Глаза его были линялые, как пальто, а~черты лица довольно расплывчатые.
Никто в~офисе не~мог точно вспомнить, как он~выглядел.
Да~что там.
Они вспоминали о~нем, если нужно было заполнить какую-то бумагу.
А~в~остальное время его словно и~не~существовало в~офисе.
Иногда он~просто смотрел в~монитор по~пять часов, и~несколько раз ему даже не~выплачивали зарплату.
Но~ему было все равно.
И~это состояние длилось уже очень давно.
 
Через две недели в~офис перевели нового начальника и~он~обратил внимание на~человека, который по~сути мешается, и~не~приносит особого дохода.
Решение пришло быстро, Дмитрий Иванович получил расчет.
Но~это никак не~мог расстроить Дмитрия Ивановича.
Его самого удивляло, что он~как-то так сделал, что начал получать пособие по~безработице и~вполне довольный жизнью он~стал жить дома, изредка выходя на~улицу за~продуктами.
 
Иногда Дмитрию Ивановичу приходилось еще выносить мусор.
Однажды он~увидел человека, уверенно копающегося в~мусорном баке.
Тот повернулся и~посмотрел на~Дмитрия Ивановича.
Он~был, наверное, единственным за~многие годы человеком, который увидел Дмитрия Ивановича.
Посмотрев на~него, он~неодобрительно хмыкнул и~в~этот же~миг Дмитрий Иванович перестал для него существовать.
Они тоже находились в~параллельных мирах, лишь изредка соприкасавшихся.
Внешне они были похожи: сутулые, словно под тяжестью нескольких тонн плечи, равнодушный взгляд, ничего не~выражающий.
Но~того человека все же~были цели и~определялись они его запахом и~тем, что он~не~ел~несколько дней.
А~Дмитрия Ивановича целей никогда не~существовало.
Он~просто плыл по~течению жизни, не~особо задумываясь над ней.
 
В~тот день в~конце октября ему совершенно нечего было делать, и~он~пошел в~единственный городской парк.
Там, сидя на~скамейке, он~наблюдал за~детьми и~их~мамами или бабушками.
Вокруг кружились голуби, с~каждым годом становившиеся все нахальнее и~нахальнее.
Они плотной толпой обступали всех, у~кого оказывалась еда, будь то~семечки или хлебные крошки, и~шумной толпой отбирали эту еду у~человека и~друг у~друга.
Потом при выходе из~парка, увидев как кошка присев на~все четыре лапы готовит тело к~прыжку, перенося свой вес вперед и~как в~предвкушении еды хищно дергается ее~хвост, он~понял, что даже с~кошкой они живут в~абсолютно разных реальностях, а~еще понял, что кошка в~этот миг была красива.
Особенно глаза, сосредоточенные лишь на~определенной цели и~не~видевшие ничто другое.
У~кошки есть цель~---~не умереть с~голоду и~дать жизнь кучке мелких котят.
Для этого она каждый раз выполняет работу, проверяя территорию, выпрашивая еду у~людей или охотясь на~голубей.
Оказалось, что у~кошки жизнь еще более непредсказуемая, чем у~людей.
 
Следующие дни текли медленно и~однообразно, ничем особенно не~выделяясь.
Но~разве это могло потревожить Дмитрия Ивановича.
Не~знаю, но~он~уже точно не~мог считаться человеком.
Потому что Человек, именно настоящий человек, это нечто большее, чем обыватель или homo sapiens sapiens.
Потому что внутри настоящего человека должно быть еще нечто, наличие которого говорит о~том, что он~человек, а~не~робот или кукла.
Это нечто есть у~каждого с~рождения.
И~это может развиваться, может остаться таким же, а~может потеряться.
Развивать это помогает труд, ведь есть же~труд не~только физический, но~и~умственный, развиваемый книгами, искусством, наукой, разговорами.
Люди веками трудятся над собой, называя это саморазвитием.
А~тех, кто собой не~занимается, зовут обывателями.
А~как же~назвать тех, кто потерял это? Людей вроде Дмитрия Ивановича? И~да, я~взяла еще щадящий пример.
Он~просто притворяется нормальным человеком.
А~есть же~те, кто, делая что-то для улучшения, и~как бы~трудясь над собой сами же~превращают себя в~кукол, у~которых нет ничего человеческого кроме тела и~лица, да~и~те, иногда настолько изуродованы, что понимаешь, нет в~них ни~капли от~людей.
 
Перемены в~жизни Дмитрия Ивановича наступили.
Да, рано или поздно такое тоже случается.
И~были эти перемены шансом на~спасение.
В~начале ноября пришло письмо от~младшего брата.
Тот, не~видя своего старшего брата в~течение десятилетия, решил встретиться, поэтому приезжал к~нему на~месяц вместе со~всей своей семьей.

~
Быстро прочитав короткое электронное письмо, Дмитрий Иванович о~нем тут же~и~забыл.
И~каково же~было его удивление, когда к~нему в~квартиру заявился его брат через пять дней.
Сначала, Дмитрий Иванович крепко испугался, когда увидел в~дверном проеме викинга или великана ростом два метра в~огромной теплой куртке и~с~не~менее огромными рюкзаками, чемоданами и~пакетами.
Следом за~ним вошла его жена миниатюрная женщина с~очень живыми и~блестящими глазами.
Она впорхнула в~квартиру, и~с~одним только ее~присутствием в~квартире стало тепло и~уютно.
Потом вбежали трое детей и~огромный золотистый ретривер.
Сразу стало шумно, громко, ярко.
В~квартире забила жизнь, и~очнувшаяся квартира была искренне этим довольна.
Брата звали Алексей Иванович, а~жену его Наталья.
Он~стал фермером, выращивал коров, поставлял мясо и~молоко в~Москву, занимался сыроварением.
К~нему приезжали сыровары из-за границы.
Они жили в~большом уютном деревянном доме рядом с~сосновым бором.
Его предприятие дало много рабочих мест в~селе, где он~решил обосноваться.
Это решило проблему с~занятостью.
Он~пользовался уважением, был общительным, добрым, умел шутить и~подбодрить.
С~собой он~привез продукты со~своей фермы и~очень довольный рассказывал о~своей жизни старшему брату.
Но~Дмитрию Ивановичу было слишком все равно на~брата и~его жизнь.
Он~не~всегда до~этого помнил, что у~него есть брат.
А~вот Алексей Иванович о~брате помнил постоянно.
Но~он~помнил того человека, с~которым вместе вырос, а~вот этого Дмитрия Ивановича.
К~тому же~Алексей Иванович не~видел брата сейчас, он~видел лишь свое представление о~брате, да~и~только.
Но~впрочем, Дмитрий Иванович был этому несказанно рад.
Он~смотрел на~своего младшего брата и~радовался тому, каким сильным и~красивым тот стал.
 
А~брат его и~вправду являл собой нечто великое.
Он~сам работал руками, сам доил, сам собирал все вначале, а~потом когда дело пошло, он~стал хорошим руководителем.
Но~на~этом дело не~заканчивалось.
Он~внедрял новые технологии, следил за~зарубежными новинками, делился опытом с~коллегами, и~они обменивались с~ним.
На~производстве можно было встретить разных не~только русских инвесторов.
Конечно, все это создавалось чуть менее двадцати лет, но~результат был грандиозный.
При этом Дмитрий Иванович следил за~новинками не~только науки, но~и~искусства.
Пожалуй, он~представлял собой идеал человека прошлого века, в~котором тело и~ум~были в~гармонии, в~равномерных пропорциях.
И~семья его была такая же, но~в~основном семья Алексея Ивановича была хлопотливым трудом его жены.

~
А~Дмитрий Иванович сидел за~праздничным столом, а~фигура его была все такая же~сгорбленная, а~глаза такие же~выцветшие.

~
Ноябрь был в~самом разгаре.
Жуткий пробирающий до~костей ветер гнал случайных прохожих по~домам.
В~то~утро Дмитрий Иванович, как и~Феня десять лет назад смотрел в~окно.
Потом что-то пробормотав, сунул ноги в~ботинки и,~оставив пальто висеть на~крючке в~коридоре, вышел на~улицу, забыв закрыть входную дверь.

~
Несчастный Алексей Иванович сразу заметил отсутствие брата и~тут же~вызвал полицию.
Потом он~возглавил отряд волонтеров, но~поиски так и~не~увенчались успехом.
Это поселило печаль в~глубине глаз Алексея Ивановича, но~ему нужно было возвращаться к~работе, поэтому через две недели он~со~всей своей семьей уехал, оставив квартиру надолго одну и~поселив в~ней холод.

~
Дмитрий Иванович долго бродил по~городу.
Пару раз натыкался на~поисковые отряды, видел свое изображение на~заборах и~фонарных столбах, видел и~усы, пририсованные ему мальчишками на~одной из~фотографий.
Он~видел своего брата, напомнившего ему Женю, но~волонтеры не~замечали его.
Он~словно не~существовал для них, а~он~был полностью равнодушен к~их~поискам и~им~самим.
Они существовали в~двух разных никоим образом не~соприкасающихся вселенных…
 
