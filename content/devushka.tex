\chap{Идеальная девушка} 

\lettrine[lines=3, loversize=0.1]{О}{}на еще раз перебрала в~уме все те~качества, которыми должна обладать идеальная девушка.
Красивая, милая (а~это не~одно и~то~же?), забавная, добрая, с~хорошим чувством юмора.
На~этом моменте Она всегда стопорилась.
Чем измеряется чувство юмора? Как понять хорошее оно или плохое? Считается ли~за~наличие этого самого чувства умение смеяться над английскими комедиями и~неутомимая любовь к~дуэту Лори--Фрай? Но~вот друзья её~смеются над совершенно другими вещами и~в~компаниях их~обычно ценят.
А~Она их~шутки не~понимает.
Ей~они кажутся плоскими, надуманными, пошлыми.
Ладно, Она где-то вычитала, что важно, чтобы чувство юмора совпадало.
Что ж...
С~этим ничего особого сделать нельзя.
Какие ещё качества определяют девушку, вслед которой оборачиваются все мужчины от~10~до~60,~и~чьей улыбки готовы добиваться все белокурые и~голубоглазые принцы со~стальными мыщцами и~не~менее стальными характерами? Ах~да, она должна быть женственной, уметь слушать, хозяйственной и~при этом легкой в~общении, самостоятельной, но~слабой, с~сильным характером, который из~неё не~будет выпирать слишком ярко, своими убеждениями, принципами, но~не~слишком строгими...
И~что-то еще.
Что-то важное Она упустила.
Ах, опять.
На~такие вещи нужно тренировать память.
А~что если записывать полезные качества девушки на~руке? Как шпоры перед важными контрольными в~школе? Нет, отдает диснеевскими мотивами.
Какая-то из~принцесс так делала.
Значит, в~реальности вряд ли~сработает.

Она покрутилась перед зеркалом раз в~пятый или двадцатый.
Недовольно наморщила нос.
Стерла помаду шоколадного оттенка.
Она сужала и~без того узкие губы, превращая их~в~дождевого червя.
Бррр.
Нужен блеск.
Вроде эффект влажных губ снова в~моде? 
А~какой оттенок лучше ягодный или натуральный розовый? 
А~не~будут ли~губы слишком яркими и~не~сделает ли~Она два акцента? 
Ужас! 
Это будет совсем уж~дурной тон.
Кто захочет общаться с~девушкой со~слишком ярким мейком,когда в~моде естественность?!

Она решила сначала определиться с~одеждой.
Пять образов.
Два платья.
Две юбки.
Шорты.
Разные футболки.
Разные прически.
Сотня фотографий перед зеркалом.
В~разных позах.
С~разным светом.
Тысяча сообщений друзьям, начинавшихся либо с~<<SOS>>, либо <<911>>.
Как итог, каждому другу пришелся по~вкусу разный образ.
В~конце концов Она вспомнила важный совет: <<Оденьтесь на~свидание по~погоде.
Нет ничего хуже синих дрожащих ног в~мини-юбке, когда на~улице $-15$ и~дует промозглый ветер.
Будьте уместной.>>
Легко сказано, однако для воплощения требуется много усилий.
Но~совет дельный.
Выглянув в~окно, Она увидела, что у~солнца, по~видимому, сегодня короткий рабочий день и~оно ушло на~час пораньше, сбежало с~ненавистной работы.
Как и~Она.
Над городом висели тучи точь-точь по~оттенку подходящие цвету ее~лица.
Дул ветер, заставлявщий брови сдвигаться все ближе и~ближе к~переносице.
Погода совершенно не~подходила под свидание.
Но~разве когда-нибудь это останавливало женщин на~благородном пути поиска своего счастья и~спутника жизни? Пришлось доставать джинсы, кросовки, сразу же~нашелся стильный обтягивающий топ, обнажающий плечи, теплая куртка с~капюшоном на~случай дождя.
Удобная сумка, в~которую влез дождевик и~деньги на~такси.
С~макияжем и~прической тоже было решено вмиг.
Топ требовал только распущенных волос, а~намечавшийся дождь обещал испортить любой макияж.
Так что водостойкая тушь, стрелки, прокрасить брови, нанести неяркий блеск, чтобы не~отвлекать внимание от~топа.
Впору благодарить дождь, что проблема выбора так легко и~безболезненно решилась.

\begin{center}
***
\end{center}

Три часа.
Три часа сборов.
А~Она идет в~итоге в~джинсах.
Се~ля~ви.
Время было рассчитано идеально.
Опоздание на~10~минут учтено.
Но~вот проблема: Она стоит в~назначенном месте в~ТЦ, а~прекрасного принца всё нет, и~смски от~прекрасного принца тоже.
Она стояла и~стояла.
А~время неумолимо быстро бежало вперед.
Сначала одна минута, потом другая и~так набежало полчаса.
Она обошла все магазины на~втором этаже.
Но,~о,~чудо! Звук уведомления.
Прекрасный принц пишет о~жутком чудовище (читай: пробке), что задержало его на~пути к~принцессе.
Но~он~борется с~ним, и~явится непременно, только вот принцессе стоит подождать.
Терпение значилось в~качествах идеальной девушки, и~она стала ждать.
Она нашла чудную лавочку и, провалившись в~хитросплетения сюжета и~интриги, что плела Бекки Шарп, спокойно и~главное с~пользой (качество идеальной девушки номер 8265921: не~упускать ни~минуты, всегда быть чем-то занятой) провела время.
И~снова чудо! Еще одна смс.
Принц победил чудовище и~ждал принцессу у~фонтана.
На~беду Она обладала хорошим зрением (не~важное, скорее даже ненужное качество для девушки) и~вот прекрасного принца и~не~увидела.
Она поняла, что три часа были потрачены впустую.
Досадное обстоятельство, но, как говорится, се~ля~ви.
Не~для тощего парнишки со~стрёмной прической и~в~застиранной синей футболке был надет модный топ, не~для него найдены самые обтягивающие из~имеющихся джинсов.
Она фыркнула, пожала плечами, печально усмехнулась и~легкой походкой вышла из~ТЦ~навстречу зажигающимся фонарям и~гудящим автомобилям.
А~на~другой стороне от~фонтана тоже стоял принц (правда белокурый и~голубоглазый), ожидающий свою принцессу.
И~вот принцесса подходила к~нему и~он~уже придумывал тысячи способов, как сбежать со~свидания.
Он~подмигнул коллеге на~другой стороне фонтана, указал на~свою принцессу и~шепнул: <<Беги!>> со~всей страстностью и~пылом, на~которые был способен.
Парнишка в~синей футболке затравленно оглянулся, ища пути отступления, и~быстро стал продвигаться к~выходу.
А~белокурый принц начал набирать номер своего друга, который смог бы~спасти его от~принцессы.

А~Она, вернувшись домой, распив с~котом чаю, села за~ноутбук и~написала в~фейсбуке, что выбирала наряд на~свидание так вдохновленно и~долго, что пропустила саму встречу.
Говорить о~том, что прождала принца больше часа было нельзя, и~что принц попался неправильный тоже нельзя, а~то~её~котировки как идеальной девушки сильно упадут и~тогда уж~точно белокурые принцы на~нее не~посмотрят.
 
\vspace{10mm}

P.~S.~Отберите у~меня женские журналы.
Они на~меня дурно влияют: характер портится, язвлю, ругаюсь.
А~чтобы быть идеальной девушкой, характер должен быть золотой.

Вот так то.

