\chap{Amour somnium breve}

\lettrine[lines=3, loversize=0.1]{О}{}на даже толком и~не~поняла, как это случилось.
Он~ворвался в~её~жизнь стремительно, легко, органично вписался в~её~ритм, в~её~окружение.
Она не~понимала этого до~того момента, пока как-то совершенно обыденно он~подошёл к~ней и~обнял сзади своими большими тёплыми руками, укутал в~своих объятьях.
А~потом он~вдруг неожиданно приподнял её~и~закружил по~комнате, затопленной уходящим августовским солнцем.
Они смеялись так громко и~радостно, что от~их~смеха в~её~маленькой квартире смеялись стены, полы, потолки, шкафы, вместе со~стоящими в~них книгами и~посудой.
Смеялась люстра, смеялись ковры и~стулья, смеялось даже солнце, и~его лучи счастливыми брызгами разбегались по~начищенному паркету.
 

В~тот день она провожала бабушку на~поезд.
Она не~хотела, чтобы та~уезжала, не~хотела отпускать её.
Бабушка всё суетилась: забывала то~паспорт, то~ключи, то~платок.
А~он~находил их, и~она злилась на~него, что он~всё находит.
А~потом они шли по~вокзалу, она плакала безрадостно и~бесшумно, но~её~рука была в~его руке.
И~это казалось самым правильным и~настоящим из~того, что есть на~свете.
Она поняла, что влюбилась.
Нечаянно, негаданно, нежданно, но~абсолютно точно.
Впервые в~жизни она поняла, что любит кого-то помимо своей семьи.
И~это было счастьем.
Неужели после стольких неудач, после множества отношений, где она испытывала лишь уважение и~нежность, она на~исходе своих двадцати умудрилась влюбиться! Невероятно, но~как же~прекрасно.
Она удивленными глазами, будто впервые увидев, смотрела на~него.
 
 --- Что? ---~спросил он
 
 --- Ничего, ---~отвечала она, спрятав улыбку на~уголках губ.
 

В~тот же~вечер, на~душной и~шумной вечеринке, она рассталась со~своим парнем, легко и~светло сказав: "Знаешь, я~наконец-то~влюбилась и~абсолютно счастлива.
Будь счастлив и~ты".
 
И~ушла к~Нему танцевать, потому что это было правильно.
А~потом гулять по~ночному городу и~целоваться в~тёмных переулках...

\vspace{10mm}

Она проснулась.
Бледно-серый ноябрьский свет больно резал по~глазам.
Она села на~кровати, мотая головой, как большая лохматая собака, силясь проснуться и~что-то понять.
В~груди разлились пока ещё необъяснимая грусть и~чувство потери.
Она помнила, что влюбилась, помнила, как это было прекрасно любить и~быть любимой.
Жаль, что влюбиться она смогла лишь во~сне.

