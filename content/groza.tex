\chap{Гроза}
\begin{flushright}
	\textit{Депрессии посвящается...}
\end{flushright}
\lettrine[lines=3, loversize=0.1]{<<П}{}ик, пик, пик…>>
Вы~никогда не~замечали, как жутко пикает касса в~торговых центрах? Она словно отсчитывает мгновения, которые тебе осталось жить.
 
 --- Да, и~еще это, пожалуйста.

Я~кладу пачку жвачек.
Орбит.
Мятные.
Чтобы отбить отвратительный металлический привкус во~рту.

В~грязной тележке эту пачку дожидаются другие товары.
Про себя медленно перечисляю их, рисуя образ каждого в~отдельности, позволяя отпечататься в~моем сознании.
Пакет яблок, пакет сока, две упаковки творога на~завтрак (жутко улыбающаяся рожица деревенской девушки--колобка на~зеленом фоне упаковки), литр кефира, крупы, сахар, разрыхлитель, зачем-то мы~взяли антистатик, и,~конечно, хлеб.
Он~сильно выделяется среди всего остального.
Ещё бы.
Не~фабричного производства, а~местного, магазинного, свежий, черный, хрустящий, с~семечками и~изюмом.
И~от~него так~восхитительно пахнет.
Он~еще теплый, прямо с~печи, и~своим теплом прогревает мою душу даже на~расстоянии.
Мне просто достаточно знать, что он~здесь, рядом.
 
 --- Не, ну~ты~слышал...
Ахахах...
Сирьёёёзно...

Голос незнакомого парня, рассказывающего своему другу крайне важную и~занимательную историю, вырывает меня из~оцепенения.
Привычным нервным жестом завожу прядь волос за~ухо.
Почему-то на~глаза опять возвращаются слёзы.
Втягиваю шею, чтобы голова оказалась вровень с~плечами, поправляя капюшон.
Беру самый большой из~двух лежащих в~тележке пакетов.

 --- Нет, нет, дорогая.
Ещё наносишься за~свою жизнь.

Она решительно отбирает пакет у~меня и~вручает свою лёгенькую сумочку взамен.
Нельзя допускать этого.
Но...
 
Вечером у~неё опять заболит спина.
А~все потому, что ей~нельзя таскать ничего тяжелее килограмма.
Но~это в~теории, а~практическая сторона жизни вещь совсем другая.
И~снова это чувство вины.
Я~ощущаю себя безмерно виноватой перед ней.
Уже очень давно.
Наверное, всегда.
И~буду твердить это вновь и~вновь.
Прости, прости, я~виновата, я~знаю.
И~еще раз прости.
Мое прости безмолвно, и~она его не~слышит, не~видит, не~чувствует.
Она пока еще не~знает, что я~так извиняюсь перед ней.
Когда-нибудь, когда время притупит раны, она поймёт.
По~крайней мере, я~очень на~это надеюсь.

Из-за этого чувства вины, я~почти не~расстаюсь с~нею.
Уже месяц я~стараюсь по~возможности все вечера проводить рядом.
Ходить с~ней за~покупками, смотреть телевизор, просто быть.
Она не~понимает в~чем дело, но~радуется безмерно.
Просто, как и~я,~старается не~говорить об~этом.
Я~знаю, вся эта ситуация тревожит ее~сильнее, чем она может и~хочет показать.
Я~практически не~общаюсь с~друзьями, про одноклассников вообще молчу.
Про друзей...~И~сказать больно.
Я~отдалилась от~них.
Сама себя убедила в~том, что это для их~же~блага, чтобы они даже не~вспомнили потом обо мне.
Но~в~глубине души я~ведь знаю.
Я~хочу побыть побольше с~ней.

Глаза покраснели снова.
Слезинка, на~минуту задержавшаяся на~реснице, была немедленно сметена ладонью.
Дура.
Так нельзя.

Мы~идем к~выходу торгового центра.
В~мутном стекле бутика напротив я~вижу наши отражения.
Два существа женского пола примерно одного роста.
Я~---~маленький пухлый карлик в~старой ветровке ярко-розового цвета.
Кричащий оттенок.
За~это время я~непростительно сильно растолстела.
Она~---~элегантная женщина средних лет, в~стильном блейзере, но~эти два огромных пластиковых пакета просто убивают весь её~вид.

На~улицу уже опустились сумерки, добавив в~атмосферу голубоватый оттенок.
Температура упала с~тридцати до~десяти градусов.
Ветер рвёт деревья.
Мы~перебегаем через дорогу на~последние секунды светофора.
В~спину нам сигналят машины.
Случайно поднимаю голову вверх, тут же~спохватившись, снова втягиваю ее~в~плечи.
Но~увиденное все еще остается со~мной.
Крупные сочные зеленые листья каштанов и~тополей вздрагивают под порывами ветра.
Ветки нещадно хлещут по~окнам рядом стоящих домов.
Вокруг мчатся машины, кричат наперерез ветру люди, а~сам он~подбавляет хаоса, кружа листовки и~прочий мусор, дуя в~лицо и~пытаясь стянуть с~нас одежду.
А~небо... Оно серо-синее, многослойное, кажущееся бесконечным.
Стальные тучи, словно крепко впаялись в~его плоть, создавая почти сюрреалистический пейзаж.
Электрическое небо.
Предгрозовое.

Мы~заходим во~двор.
И~тишина и~спокойствие этого места почти оглушают меня.
В~глубине двора на~лавочке, скрытой каштанами самозабвенно целуется парочка.
Заметив нас, они мгновенно отскакивают друг от~друга.
Девушка одаривает меня высокомерным чуть испуганным взглядом.
Сейчас она чувствует себя на~вершине мира, словно ей~открылась неведомая остальным тайна.
А~мы, простые смертные, пока не~узнали такого.
Она уверена, что хозяйка вселенной.
Как я~её~понимаю.
Когда-то я~тоже чувствовала себя королевой.
Я~отворачиваюсь и~прохожу дальше.

В~двери скрипят ключи.
В~нос из~раскрытой двери ударяет кислый запах старого дерева.
Это наш сорокалетний паркет.
Мама что-то произносит детским жеманным голоском.
Ей~кажется это очень смешным.
Потом она комментирует чьи-то действия, подшучивая над их~глупостью.
Я~угукаю в~ответ, не~особо вслушиваясь и~стараясь скрыть раздражение.
Не~люблю, когда она ведет себя так.
Люди разные.
Не~всем же~быть такими правильными, умными и~сообразительными как она.
Люди вообще по~природе слабые недалекие существа.
И~снова это чувство вины.
Нельзя так говорить о~ней.
Нельзя злиться на~нее.

\begin{center}
	***
\end{center}

Стемнело совсем.
Десять вечера.
Бог мой, а~пришли мы~только в~восемь.
Неужели прошло два часа? Я~разобрала вещи, переоделась и~села на~кровать на~секунду, чтобы посмотреть в~окно, а~прошло два часа.
Всхлип произвольно вырывается из~горла.
Тут же~ладонями зажимаю рот.
Нельзя, чтобы он~услышала.
Два часа.
Я~ничего не~помню.
И~не~помню, когда это началось.
Что-то темное забирает меня, и~я~не~в~силах с~этим бороться.
Моменты света так редки.
Поэтому я~провожу их~с~ней.
Случается, что я~на~секунду задумаюсь над чем-то, а~оказывается, что прошел уже день.
И~я~опять опоздала и~не~пришла в~школу.
Она не~знает: каждый раз я~вру ей, и~стена становится все выше, а~вина все больше.

\begin{center}
	***
\end{center}

Я~проснулась среди ночи.
Теперь я~постоянно встаю по~ночам: ведь уже давно потеряла чувство времени.
Мы~спим с~открытыми окнами.
Вечернее напряжение разразилось грозой.
Крупные ливневые капли бьют со~всей злостью, на~которую способны, по~ни~в~чем не~повинному асфальту.
Я~иду на~кухню, включаю кран для фильтрованной воды.
Набираю полный стакан.
Медленно пью, хотя не~хочется, до~спазмов в~желудке.
Окно на~кухне низкое, почти французское, этаж семнадцатый.
Завороженно подхожу к~нему, нервными подрагивающими пальцами снимаю сетку.
Дождь врывается в~прямоугольник кухни.
Капли падают на~белую ночнушку, оставляя серые разводы.
Оглядываюсь назад, словно прощаясь.
В~память врезаются красные цифры на~микроволновке, показывающие время.
\begin{center}
\texttt{2:30}
\end{center}
Бьет молния.
Синяя, с~белыми краями в~черном небе.
Дождь заливает глаза.
С~последующим раскатом грома все заканчивается.


\vspace{5mm}


Она проснулась от~внезапного сильного порыва ветра за~пять минут до~рассвета.
\begin{center}
\texttt{4:55}
\end{center}
Повинуясь интуиции, пошла на~кухню.
Сетка на~окне сорвана, пол залит водой.
Она еще ничего не~поняла, и~не~скоро поймет.

