\chap{***
} 
Подъем в~6:30, впрочем как и~всегда вот уже почти год.
Быстрый взгляд в~телефон, погладить сонного кота.
Он~опять презрительно прищурит свои жёлтые глаза и~надменно отвернётся.
Затем пройти на~кухню и,~пока пьёшь стакан холодной воды, выстоявшей ночь в~серебряном кубке (не спрашивайте даже зачем это делать, очередной параноидальный родительский всплеск), любоваться аккуратными золотистыми прямоугольниками солнца, разукрасившими кафельный кухонный пол.
Часы показывают уже 6:45, а~значит, следует поспешить.

И~вот ты~уже выходишь из~квартиры.
На~ногах старые кроссовки, потому что новые остались в~общаге в~Москве, такие же~старые лосины, в~которых только на~дачу и~ездить, сверху старая ветровка, чёрная шапка… Но, кажется, чего-то не~хватает, чего-то ставшего столь важным, что без него ты~точно никогда не~выйдешь из~дома.
Да, точно, на~руки одноразовые перчатки, на~нос многоразовую маску с~забавными пузатыми котятами, которую заботливо сшила одна из~маминых пациенток.
Конечно, доктору ни~в~коем случае нельзя болеть, ~иначе… Впрочем не~стоит о~грустном.

Ты~уже на~улице, вдыхаешь через ткань тёплый ласковый апрельский воздух.
Будто издеваясь, весна пришла к~нам именно сейчас и~развернулась в~полную силу.
Солнце кокетливо светит, прячась за~ветвями деревьев, словно юная скромница.
Почки робко, но~неумолимо пробиваются, радуя глаз своей свежей зеленью.
Ты~останавливается на~миг и~вдыхаешь и~вдыхаешь этот чудный воздух, который приходит каждый год и~каждый раз даёт надежду, и~силы, и~веру в~лучшее.
Ты~никак не~можешь им~надышаться и,~кажется, что тебе все мало, что ты~что-то недополучаешь, а~ведь так и~есть.
Маска не~даёт тебе в~полной мере заполнить свои лёгкие кислородом, из-за нее ты~в~полной мере не~можешь ощутить дыхание прохладного апрельского ветра, в~душе начинает подниматься обида.
Как? И~тебя уже и~этого лишили? Разве мало было им~твоего привычного образа жизни, планов на~лето? Но~кто такие эти непонятые мы, кто виноват во~всем случившемся? Да, никто, и~ты~прекрасно это понимаешь, никто ни~в~чем не~виноват, так бывает.
Это жизнь.
А~маски --- они нужны.
И~разве это не~честно? Впервые люди не~скрывают, что выходя из~дома, надевают маски…
 

~
Полчаса пробежки в~парке, и~ты~снова в~родном подъезде.
Подняться на~нужный этаж.
Открыть квартиру.
Снять перчатки, сложить в~специальный пакет в~коридоре, который при следующем походе на~улицу выкинут отдельно, снять ветровку, шапку, сразу же~пройти в~ванную, вымыть руки антибактериальным мылом, там же~снять маску и~в~тот же~миг постирать ее~с~антибактериальным мылом.
Вернуться в~коридор, достать отпариватель и~обработать верхнюю одежду и~обувь.
Фух, вроде ничего не~забыла? Нет, ничего.
Теперь быстрый душ, завтрак на~скорую руку, доделать дз, а~времени все меньше и~меньше, пары начинаются в~девять.
 

~
В~16:50 ты, наконец, оторвешь свой взгляд от~ноутбука, глаза будут нестерпимо болеть и~слезиться, голова будет раскалываться на~части, кажется, снова поднимается температура.
И~нет, это не~потому что ты, несмотря на~все принятые меры предосторожности, умудрилась-таки им~заразиться, нет, это реакция твоего организма на~восьмичасовой учебный день перед ноутбуком.
Ты~поднимаешь голову из-под горы учебников, которой оказалась завалена, мутным взором обводишь комнату, не~до~конца понимая, что же~с~ней случилось за~8~часов.
Находишь в~эпицентре разгрома несчастного кота, который пытался хоть как-то до~тебя дозваться, и~понимаешь, что уголки твоих губ немного приподнялись.

Обычно тебе еще нужно сходить в~магазин или в~аптеку, чтобы купить продукты и~лекарства своим бабушке и~дедушке, а~также некоторым соседям, которым уже больше 65.
Но~сегодня особенный день, сегодня никому ничего не~надо приносить.
Но~почему-то ты~идешь к~зеркалу и~впервые за~полтора месяца начинаешь краситься.
И~плевать, что за~эти полтора месяца кожа отдохнула от~макияжа, плевать на~то, что за~маской все равно никто ничего не~увидит.
Вся твоя душа, вся твоя внутренняя суть требует этих действий.
Накраситься, надеть легкомысленную светлую блузку, длинную зеленую юбку, которую ты~так и~не~успела выгулять, подобрать к~образу украшения, уложить волосы, достать черные башмачки, не~забыть, конечно, про маску, которую ты~тоже подбираешь под образ, перчатки в~тон маске --- вуаля, ты~готова к~прогулке.

Нет, не~подумайте, ты~не~собираешься ничего нарушать, ты~идешь в~парк, который прямо через дорогу от~твоего дома.
Но~впервые за~полтора месяца тебе это нужно, небольшой глоток свободы посреди неразберихи, запретов и~скрытой паники.
За~сборами не~замечаешь, как быстро прошло время и~что на~город уже начали опускаться сумерки.
В~легкой синей дымке ты~доходишь до~парка, вставляешь наушники и~погружаешься в~собственный прекрасный мир, мир, где нет пандемии, где нет масок и~антисептиков, где властвует лишь магия сумерек.
Ты~оказываешься наедине с~самим собой и~впервые можешь спокойно думать об~абстрактных понятиях и~вещах.
Завтра ты~снова погрузишься в~учебу, снова будешь мечтать об~окончании периода самоизоляции.
Но~сегодня ты~позволяешь себе жить этим самым мигом, этим весенним воздух, обманчиво обещающим перемены, и~это то, что коронавирус у~тебя не~отнимет.

И~вот ты~идешь, слушая, как в~динамиках Боб Дилан выпевает: «~Knock-Knocking on~Heaven s~door …»,~---~ты~идешь навстречу разгорающемуся закату, не~замечая, как за~тобой один за~другим вспыхивают фонари.
 
