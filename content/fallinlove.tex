\chap{When you fall in~love...} 

\lettrine[lines=3, loversize=0.1]{У}{}тро понедельника, как обычно, не~предвещало ничего хорошего.
Это явление так~же~неизменно, как и~то, что на~Киевской в~полдевятого будет настоящая пробка из~людей.
Видимо, потому даже Меркурий в~этот день становится ретроградным и~луна уходит в~пятый дом солнца.
Все боятся понедельника, тихо его ненавидят, но~почему-то жить без него не~могут, не~меняют каждодневную рутину и~с~наступлением новой недели снова рвутся с~жизнью в~бой за~свой кусок счастья.
На~самом деле я~преувеличиваю, нет даже хуже, гиперболизирую.
Если спуститься с~утра пораньше в~метро, например в~Москве, то~можно увидеть столько оттенков безразличия и~усталости, сколько и~представить себе сложно.

Я~безбожно опаздывала, причем начала опаздывать еще на~оранжевой ветке, когда поезд решил постоять а~тоннеле, поэтому на~кольце ворвалась в~вагон со~всей решительностью и~всем отчаянием, на~которые только была способна и~в~первые несколько секунд, прижатая со~всех сторон к~людской массе, задвинутая напирающей толпой к~двери, я~никак не~могла отдышаться ~и~осознать, что пока всё, погоня за~временем закончена.


А~потом я~увидела его…
 
Говорят, что любви с~первого взгляда не~существует, что это банальная симпатия, которую романтические личности склонны преувеличивать и~придавать ей~некий флёр чувственности и~загадочности.
И~я~в~принципе с~этим привыкла соглашаться, ведь даже для формирования влюбленности нужно не~только физическое притяжение и~внешняя привлекательность друг для друга, но~и~притяжение характеров, интересов...
А~для этого человека требуется хоть немного, но~узнать, поговорить с~ним и~уже по~манере речи осознать, сможешь ли~ты~в~него влюбиться или нет.
Но, как обычно, было в~этот день одно большое но, хотя, конечно, я~сама не~берусь дать какое-то определение тому чувству, от~которого вмиг потеплело в~сердце.
 
Стремясь стать поудобнее в~вагоне или хотя бы~занять сколько-нибудь устойчивое положение, я~медленно разворачивалась и~случайно сумкой задела молодого парня.
И,~конечно, я~решила тут же~извиниться, и~на~этом можно ставить точку и~заканчивать рассказ.
 
И~даже сейчас по~прошествии нескольких дней я~помню каждую деталь, секунду, миг, жест.
И~вот ты~поднимаешь голову, ваши глаза встречаются, у~тебя учащается пульс, расширяются зрачки, и~ты, распознав симптомы и~их~испугавшись, резко и~нелепо прячешь лицо в~волосах и~огромном шарфе.
Но~улыбка, улыбка против воли появляется на~губах и~легкими смешинками прячется в~глазах, а~подрагивающие ресницы и~покрасневщие от~смущения щеки выдают тебя с~головой.
Ты~снова чувствуешь себя как в~младших классах, когда тебя переполняют чувства, а~ты~и~признаться не~можешь и~боишься, потому что все вокруг засмеют тебя.
Но~в~этот раз все по-другому: каким-то пятым чувством ты~замечаешь его улыбку и~с~огромной боязнью и~надеждой снова поднимаешь глаза, чтобы посмотреть уже глазами на~эту улыбку, видишь ещё и~его отчаянно задранную вверх бровь, внутри тебя что-то взрывается и~вмиг становится всё легко, понятно и~просто.
А~когда ваши взгляды вновь встречаются, вас обоих пробирает такое веселье и~смех, и~улыбки уже не~остановить.
Вы~оба всматриваетесь в~каждую черточку, подмечаете малейшие детали (легкий пушок на~щеках у~нее, родинку над бровью у~него) и~впитываете, впитываете в~себя эти детали, стараетесь затвердить на~подкорке каждую черту в~доли секунды ставшего таким родным и~любимым лица.
Кажется, что даже время остановилось и~не~осталось ничего вокруг в~этом мире, кроме вас.
И~плевать, что вы~зажаты в~плотном кольце безразличных друг другу и~самим себе людей, плевать на~всё, кроме этого непрошенного чувства, подобного чуду.
Вам двоим кажется, что прошло несколько часов, а~то~и~дней, хотя в~реальности всё заняло 3~минуты, ровно одну остановку метро от~станции Октябрьская до~Парка Культуры.
А~затем...


{\tt

\vspace{5mm}
<<Станция парк культуры, переход на~сокольническую линию.
This is~park kultury...>>
\vspace{5mm}
}


И~все~---~вы~снова чужие люди.
Вы~расходитесь по~своим делам и~жизням, чтобы никогда вновь не~увидеться.

И~не~нужно тешить себя надеждами, о~том, что вы~рано или поздно встретитесь снова.
Не~встретитесь: жизнь ---~это не~американская мелодрама с~обязательным хэппи эндом.
И~нужен ли~вообще этот счастливый конец? А~если и~встретитесь, то~вряд ли~узнаете друг друга, черты лица уже сейчас расплываются перед глазами, вы~помните только мелкие детали, а~общую картинку никак не~получается собрать.
И~тем более не~стоит из-за этого расстраиваться.
Ведь у~вас каждого за~спиной отношения, привычные, знакомые, где всё, как на~ТО~(проверка движка, замена топлива), но~вам с~ними комфортно и~вряд ли~бы~вы~согласились их~променять на~что-то новое, незнакомое... 
~
Да~вы~и~имён друг друга не~знаете, да~и~не~нужны они.
Но, расходясь по~своим делам, улыбка будет освещать ваше лицо и~настроение будет просто заоблачным, хоть день, хоть час, хоть пять минут.
И~во~всем вокруг: и~в~лужах, и~в~еще не~погасших фонарях, и~в~сером небе, и~зданиях, в~действиях и~мыслях будет невероятная, непомерная легкость...

В~Москву пришла весна.