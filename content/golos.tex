\chap{Голос Родины} 
\lettrine[lines=3, loversize=0.1]{К}{}огда на~земле рождается человек, он~получает от~природы или от~Бога, не~знаю от~кого точно, дар, чудесный дар, называющийся родиной.
Родина --- это место, куда он~всегда сможет вернуться, чтобы ни~случилось, и~где его всегда будут ждать.
Но, как и~всякий дар, родина имеет свои особенности, которые нужно принять и~понять.
Можно сказать, что мы~родину не~выбираем, потому~что это она выбирает нас.
Но~случается, что между человеком и~родиной возникают непонимания, из-за~которых люди уходят из~дому и~скитаются потом по~свету одинокие, так и~не~нашедшие своего места в~мире.
\begin{center}
***
\end{center}



Большая капля упала на~стекло и~плавно потекла вниз, но~не~по~прямой, что было бы~более логичным, а~сделав странный крюк.
И~все равно попала в~то~место, где оказалась бы, если бы~текла по~прямой.
Так может в~этом и~кроется странная загадка жизни? Какие бы~мы~вихры не~совершали, что бы~ни~вытворяли, мы~все равно будем там, где нам уготовано судьбой.
Как эта капля.
Дэн поднял руку и~протер запотевшее стекло.
Он~улыбнулся капле, которая завершив свой путь, слилась с~другой каплей и~теперь они вместе приближались к~оконной раме, но~теперь уже вместе.
Потому, что если его догадка правильна, то~еще не~все потеряно, и~рано или поздно он~найдет свое место в~мире.
То~самое место, которое ему уготовила судьба.
А~долго ли~он~бегал от~судьбы?
 
Все началось еще в~раннем детстве, когда он~еще не~был Дэном.
Эту фразу он~впервые услышал от~отца, и~если бы~так ничему и~не~научился за~эти годы, то~винил бы~его в~своей несостоявшейся жизни.
\textit{<<Человек всегда связан с~тем местом, в~котором родился.>>}
Эта связь, столь незаметная на~первый взгляд, на~самом деле соединила человека и~его родину так сильно, что подчас не~понять, где есть человек, а~где его родина.
Разные вариации этой фразы Дэн, бывший в~то~время просто Денисом, слышал потом на~протяжении всей своей жизни.
Она словно преследовала его постоянно, не~давала покоя ни~в~школе, ни~дома, ни~на~встречах с~друзьями, заставляя каждый раз сомневаться в~себе и~задумываться, я~ли~это говорю или во~мне звучит моя родина.
Денис искал следы этой связи в~каждом человеке, с~которым начинал близко общаться.
И~именно, она стала решающим фактором в~его побеге из~дома.
Денис хотел понять, где он, а~где родина, и~сколько в~нем его самого.
Поэтому решил, что лучший выход --- это побыть вдали от~родины некоторое время, примерно года два.
Тогда Денис считал, что два года, проведенные в~другой стране, вдали от~всего привычного, помогут ему найти себя, но~он~не~понимал, что на~поиски своего я,~некоторые люди могут потратить всю жизнь, но~так и~не~найти ответ, а~другие с~самого детства знают, кто они и~каково их~место в~мире.

Тогда, в~семнадцать лет, когда ты~видишь только одну цель, когда твои суждения категоричны и~когда ты~уверен, что мир только и~ждёт твоих действий, словом в~то~время, когда человек слеп сильнее всего, Денис сбежал из~дома и~стал Дэном.
Он~только закончил школу и,~как и~все, подавал документы на~поступление в~Вуз.
Нет, мысль о~побеге не~была спонтанной, как и~сам его побег.
Свои планы он~тщательно обдумывал на~протяжении пяти лет, а~к~побегу готовился целый год, предварительно подсчитав все расходы, проблемы с~визами, границами и~многим другим.
В~интернете он~познакомился с~людьми, готовыми предоставить ему кров на~первое время.
В~его планы входило объездить всю Европу за~два года, не~задерживаясь ни~в~одном месте более чем на~месяц.
Одного он~не~учел: страданий близких, и~целого года вранья родителям в~глаза, хотя врать Денис никогда не~любил.
И~в~то~время, когда родители были уверены, что их~сын готовится к~вступительным экзаменам, он~взял деньги, которые копил в~течении пяти лет, вещи, которые могут понадобиться в~первое время, сел на~поезд и~больше дома не~появился.

Денис не~боялся ни~языкового барьера, он~с~детства учил два иностранных языка, ни~воришек, брать у~него было особо нечего, ни~полиции, потому что он~был просто путешественником.
Денис не~планировал забрасывать учебу, он~нашел себе какую-то заочную школу и~много читал.
Читал пока ехал в~поезде, самолетов он~избегал, читал в~кафе, на~улице, на~заправках и~автостоянках, ночью и~днем, в~автобусах --- везде и~всегда, когда появлялась такая возможность.
Первый год прошел очень легко.
Он~объездил половину Европы, познакомился с~разными людьми, улучшил знания языков, не~нуждался в~деньгах, потому что в~том городе, в~котором останавливался, всегда находил нетрудную работу.
Во~второй год было чуть тяжелее.
Денис перестал видеть смысл своей поездки.
Он~начал понимать, что поступил глупо, что стоило подождать и~поступить в~университет, потом отработать несколько лет, а~дальше он~был~бы~хозяином своей жизни.
Но~нет, молодости не~свойственно быть терпеливой.
А~сейчас отступить --- значило бы~сдаться, признать свое поражение, поэтому решено было хотя~бы~одно дело довести до~конца.
В~конце второго года, подъезжая к~самой границе, Денис осознал, что дома он~в~принципе никому не~нужен, что он~упустил свое время, бездарно истратил на~глупые поиски, потому что так и~не~приблизился к~ответу, и~родителям принесет только дополнительное огорчение.

Теперь он~думал о~родителях.
С~того дня он~не~переставая думал о~родителях каждый день.
А~ещё он~понял, что это --- расплата за~пренебрежение родиной, за~нежелание принять свой дар и~глупые эгоистичные настроения.
Денис, вернее уже Дэн, отправил себя в~добровольное изгнание.
В~тот день дождь шел плотной стеной, а~Дэн оставил свой зонт в~хостеле, и~пытался рюкзаком прикрыть голову, но~оставил это и~позволил каплям свободно падать на~его тело.
В~тот миг все смешалось, предметы потеряли привычные границы, не~было понятно, где небо, где земля, железнодорожный вокзал, словно перестал существовать, люди исчезли, исчезли их~голоса и~шум, производимый ими, остался только звук дождя и~ничего больше.
Дэн перестал понимать, кто он~и~что здесь делает, он~не~мог отделить одну мысль от~другой и~чувства от~мыслей, он~перестал осознавать себя --- миг, и~помешательство прошло.
Он~снова был Дэном, снова стоял на~вокзале, только теперь в~его голове поселился странный шум, который с~тех пор его не~покидал.

Дэн научился жить с~шумом в~своей голове, смирился с~ним, как жители больших городов смиряются с~тем, что с~приходом ночи, город не~засыпает.
После этого Дэн продолжил свое путешествие, только уже не~заезжал в~крупные города и~туристические центры, где рисковал наткнуться на~соотечественников.
Так он~и~стал жить: автостопом доезжал до~очередного маленького городка или деревушки, рядом с~которой была автозаправка.
Там нанимался эту самую автозаправку, кем угодно, оставался в~этом городке на~месяц, изучая местных жителей.
С~годами он~научился легко понимать людей, и~использовать свое умение себе на~пользу.
Теперь он~спал, где придется, ел, что пошлет Бог, или не~ел~вообще, отрастил бороду, не~смотрел в~зеркала, редко менял одежду, зато почти все деньги тратил на~книги, которые отсылал друзьям, в~надежде, что от~него в~этой жизни после смерти останется хотя бы~маленькая библиотека.

На~маленькой автозаправке, около такого же~маленького городка в~центре Европы он~оказался так~же, как и~на~всех своих остальных местах работы.
Вообще, он~питал странную слабость именно к~заправкам, возможно, потому что, заправки --- неотъемлемая часть любого путешествия, пускай и~быстро забываемая.
Жил он~здесь же, на~заправке, спал на~раскладушке, ел~чипсы и~запивал их~минеральной водой или пепси.
По~правде говоря, он~должен был покинуть это место еще пять месяцев назад, но~что-то не~давало ему этого сделать.
Дэн снова протер окно, его взгляд скользнул на~календарь.
Смешно, восемь лет, именно в~этот день он~стоял на~вокзале и~тогда тоже шел дождь.
То~есть, уже десять лет он~не~был дома, не~слышал родную речь.
Десять лет --- большой срок.
Если тогда было поздно возвращаться с~покаянием, то~уж~сейчас тем более.

Внезапно Дэн решил, что эту дату стоит отметить.
Он~быстро собрал свои пожитки, благо за~эти годы их~накопилось немного, и~он~разучился привязываться к~вещам.
К~заправке подъехал автомобиль, водитель направлялся в~крупный город, располагавшийся неподалеку, и~согласился его до~туда подбросить.
Они ехали два часа, и~все это время шел дождь.
Это не~был ливень, а~скорее тот дождь, под который хочется залезть в~плед и~грустить или включить какую-нибудь грустную музыку, привалиться к~автомобильному стеклу и~ни~о~чем не~думать.
Водитель высадил Дэна около какой-то кафешки.
Всю дорогу шум в~голове усиливался, и~сейчас Дэн был немного дезориентирован.
Он, словно сомнамбула, вошел в~кафе, сел за~дальний столик, не~до~конца понимая, зачем он~это делает.
У~уставшей официантки попросил чаю и~снова уставился на~стол, будто надеялся, что там найдет ответы.
Но~стол молчал.
Его шероховатая поверхность когда-то была гладкой и~блестящей, но~сотни посетителей и~чашек оставили на~нем следы, и~теперь он~был погружен в~себя, не~замечая человека, ждущего ответа от~него.

А~посетители приходили и~приходили, чай наливался, ароматный кофе готовился, и~уставшая официантка носилась между всем этим, стараясь угодить всем.
Она тоже была погружена в~себя, и~часть ее~переживаний отпечаталась у~нее на~лице.
Вдруг у~нее зазвонил телефон, и~все, кто пришел в~кафе, обернулись к~ней, а~она даже не~сразу заметила звонок.
Потом смутилась, провела пальцем по~сенсорной клавиатуре и~быстро что-то начала говорить.
Дэн словно очнулся ото~сна.
Девушка говорила на~его родном языке, причем правильно, без~акцента.
Слышно было, что это её~родной язык.
Она врала маме, что хорошо устроилась, что спит, сколько положено, питается правильно и~соблюдает диету.
Дэн присмотрелся к~девушке.
В~ее~интонациях было что-то успокаивающее, как у~мамы, когда он~болел в~детстве.
Её~улыбка была такой же~искренней, как у~сестры, а~волосы были так же убраны в~пучок, как у~девушки, которую он~любил.
Весь вечер он~просидел в~этом кафе и~наблюдал за~девушкой.
Вернее, его глаза следили за~её~фигурой, легко лавирующей между столиков, а~мысли были очень далеко.

Он~вспоминал.
Вспоминал всё: маленькую квартиру в~центре, пробки по~утрам в~школу, саму школу с~её~строгими завучами, формой, учителями, тайными прогуливаниями уроков, выговорами родителей; поездки за~город, на~дачу, в~лес.
Он~вспомнил своё местечко на~крыше и~то, как ночами уходил туда.
Вспомнил, как готовился к~экзаменам, вспомнил, как это было важно.
Казалось, не~было этих десяти лет.
Они слились в~одно пятно, где не~было ничего, ни~тьмы, ни~света, просто большое серое пятно в~его жизни.
В~этот момент он~понял, что шум в~его голове прекратился.
Теперь там звучал голос.
Голос этой девушки, только это не~она говорила с~ним.
Он~прекрасно понял, что это голос родины.
А~это значило только одно: он~прощён.

Выходя из~кафе, он~подошёл к~официантке и~шепнул ей~на~ухо: «Спасибо».
На~улице уже стояла ночь, дождь прекратился.
За~яркими вывесками не~видно было звёзд, но~он~знал, что они там есть, точно также, как и~знал, что дома его ждут.
А~раз его простили, раз она заговорила с~ним, значит он~нужен там, значит он~возвращается домой.

Голос звучал в~его всю ночь, пока он~гулял по~городу, прощаясь с~ним, с~этой страницей своей жизни.
Денис не~понимал, что голос ему говорит, достаточно было и~того, что голос просто звучал в~его голове.
Рассвет Денис встретил на~выходе из~города.