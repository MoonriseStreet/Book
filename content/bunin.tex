\chap{И.~А.~Бунин} 
\lettrine[lines=3, loversize=0.1]{В}{} июне 1945~года в~Булонском лесу прогуливался старый человек.
Он~шел медленно, погруженный в~свои мысли, часто останавливаясь для~передышки.
Молодые парочки, поглощенные собой, казалось, не~замечали его.
А~вот его старческие глаза, видевшие и~революцию, и~войну, подмечали каждую деталь.
В~глубине души он~завидовал им, свежести их~чувств, их~легкости, их~мечтам.
Несмотря на~то, что со~дня освобождения Парижа не~прошло и~года, и~многое, из~того, что было в~старом Париже безвозвратно уничтожено войной, в~город возвращалась атмосфера легкости и~праздности.
Париж снова становился городом любви, для которой он~был слишком стар.
Да, Париж уже не~для него.
Этот город для тех, кто еще верит в~любовь.
А~он~давно еще потерял всякую надежду найти того, с~кем не~будет так одинок.
Что ему делать в~городе молодых? Он~семидесятипятилетний старик.
Ему положено уже лежать в~земле, а~не~мечтать о~любви.
Всего было в~его жизни много, но~только не~любви.
Если бы~вернуться в~прошлое и~всё исправить! 
Но~это невозможно.
А~впрочем, что же~он~собрался исправлять? 
Нет, таким как он~положено лежать в~постели у~себя дома, и~предаваться воспоминаниям.
Но~где его дом? 
Дом находится за~гранью настоящего.
Но, может, сейчас как раз время для исправления ошибок.
Пора, когда блудный сын возвращается домой.
Может, стоит забыть обиды, и~теперь когда ему уже не~страшна советская власть, посмотреть во~что превратилась Россия.
Ах, как бы~ему хотелось объездить всю страну! 
К~сожалению, Россия слишком большая, на~всю у~него не~хватит сил.
Но~если была бы~возможность вернуться, какое место он~бы~навестил первым…

Улица Большая Дворянская~1, сейчас правда называется проспектом революции.
Фу, дурацкое название.
И~зачем большевикам понадобилось переименовывать улицы? 
Улица Большая Дворянская, понятно, что на~ней живут дворяне, а~теперь проспект революции!
Неужто революция по~нему шагала? 
Нет, революция шагала по~переходам дворцов, кралась по~их~коридорам, обитым зеленым шелком.
А~дом все же~был хорошим.
Его в~1865~году купила бывшая губернская секретарша и~начала сдавать часть комнат, там мы~и~жили, недолго, конечно.
Потом семья переехала в~Орловскую область, в~имение~Озёрки.
Он~вспомнил тот день, когда впервые расстался с~домом надолго.
Он,~одинадцатилетний щуплый юноша, поступил в~Елецкую уездную гимназию.
Эта мысль тогда доставляла радость.
Но~по~просшествии четырех лет его мнение кардинально изменилось.
Вспомнил и~те~памятные зимние каникулы, когда так и~не~смог доказать родителям, что он~уже вполне взрослая и~сформировавшаяся личность, отказавшись возвращаться в~гимназию.
Вспомнил и~удивление, написанное на~их~лицах.
Тогда же~он~отдалился от~них, потому, что они не~приняли его всерьёз и~после зимних каникул он~вернулся в~гимназию.
Но~через год Юлий взял над ним опеку, и~можно было забыть об~учителях и~распорядке дня.
Тем более, учась дома, он~узнавал гораздо больше.
А~через год дебют в~печати, а~ему всего семнадцать, но~фурора он~не~произвел.
Дальше были два~тихих года.
Но~вот в~девятнадцать он~переезжает в~Орёл, идёт работать в~местную газету.
А~потом была Варенька.
Варвара Пащенко, сотрудница этой газеты.
Ах, как были возмущены, удивлены и~оскорблены в~своих лучших чувствах его родные.
Странно лишь то, что этот поступок они все же~заметили.
Чтобы как-то избавиться от~их~назойливости пришлось в~189, да, именно в~этом году, с~Варенькой переехать в~Полтаву.
Варя была одним из~его первых и~продолжительных романов.
Сейчас это называют омерзительным словом связь.
Но~она быстро забылась.
Да~и~как могла не~померкнуть какая-то Варенька перед Анной.
Анной Цакни, дочерью революционера-народника, богатого одесского грека Николая Петровича Цакни.
Но~она, как и~всякий ребенок богатых родителей, была непозволительно красива и~избалована.
Знаете, говорят, что дети рождаются от~большой любви.
Значит мы~с~ней не~любили друг друга, учитывая, что единственный ребенок, которого она мне подарила умер в~пятилетнем возрасте.
Да~и~\textit{qui se~marie par amour a~bonne nuits et~mauvais jours\addfootnote{Кто женится по любви, тот имеет хорошие ночи и скверные дни \textit{(фр.)}.}}.
Но~не~могу сказать, чтобы после развода я~сильно горевал.
Меня спасла Вера.
Вера Муромцева, её~знаменитый дядя был председателем Государственной Думы Российской империи первого созыва.
Потом что-то сподвигло меня начать путешествовать, но~ездил я~недолго, всего два года, зато посетил сразу три страны: Палестину, Сирию и~Египет.
В~1909~году мне снова дали Пушкинскую премию, а~после избрали почетным академиком Санкт-Петербургской академии наук.
Многие критики стали говорить, что я~непозволительно загордился и~стал относиться к~молодежи с~недестойным высокомерием, но~я~за~собой такого не~наблюдал…
 
А~затем наступила эта революция.
Вы~только представьте: была Россия, был великий, ломившийся от~скарба дом, населенный могучим семейством, созданный трудами многих и~многих поколений, освященный богопочитанием, памятью о~прошлом и~всем тем, что называется культом и~культурой.
Что~же~с~ним сделали? 
А~появился выродок, нравственный идиот от~рождения, Ленин явил миру как раз в~разгар своей деятельности нечто чудовищное, потрясающее, он~разорил величайшую в~мире страну и~убил миллионы людей, а~среди бела дня спорят: благодетель он~человечества или нет? 
Низменней, лживей, злей и~деспотичней этой деятельности еще не~было в~человеческой истории даже в~самые подлые и~кровавые времена.
~Я~до~последнего тянул с~отъездом из~России.
Но~в~1920~покинуть дом все же~пришлось.
Большевики подходили к~Одессе.
Я,~как и~многие мои соотечественники, переехал во~Францию.
С~тех пор постоянно меняю жилье.
Не~могу по~долгу оставаться на~одном месте.
В~1933~году меня сделали лауреатом Нобелевской премии, дали 120~тысяч франков.
Я~нуждался в~них, но~были те, кто нуждался сильнее меня.
В~1939~додумался снять виллу «Жаннет» В~Грасе.
Жена была со~мной.
Знаете, никогда не~был сторонником \textit{L'eau gate le~vin comme la~charette le~chemin et~la~femme l'ame\addfootnote{Вода портит вино так же, как повозка дорогу, и как женщина душу\textit{(фр.)}.}}.
Она помогала мне прятать трёх евреев.
Есть поступки, которыми большего всего гордишься в~жизни.
Так я~горжусь тем, что не~продался фашистам, не~стал сотрудничать с~ними.
Не~понимаю, как могли Гиппиус, Мережковский поддаться им.
Как можно так ненавидеть родную страну…

Но~что-то я~о~грустном.
Это уже в~прошлом, хотя было совсем недавно.
А~как все-таки хочется поехать, посмотреть, побывать в~знакомых местах, но… 
Поздно, поздно… Я~уже стар, и~друзей никого в~живых не~осталось.
Из~близких друзей остался один Телешов, да~и~тот, боюсь, как бы~не~помер, пока приеду.
Боюсь почувствовать себя в~пустоте.
Франция стала для меня второй родиной.
Я~привязался к~Франции, очень привык, и~мне было бы~трудно от~нее отвыкать.
А~брать паспорт и~не~ехать, оставаться здесь с~советским паспортом --- зачем же~брать паспорт, если не~ехать?
Раз я~не~еду, буду жить, как жил, дело ведь не~в~моих документах, а~в~моих чувствах… 
