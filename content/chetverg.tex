\chap{Чистый четверг}

\lettrine[lines=3, loversize=0.1]{В}{}есна в~этом году была ранняя.
Как-то слишком быстро и~незаметно растаял весь снег в~парке.
И~к~началу марта можно было уже спокойно ходить в~тонкой куртке, без сменной обуви и~шапки в~школу.
Все мои надежды на~то, что как и~в~том году на~тридцать первое марта пойдет снег, и~серые унылые улицы нашего провинциального городка сказочно преобразятся, как в~начале осени, пошли прахом.
Снег и~не~думал выпадать.
Более того, погода стояла преотличная: все дни светило солнце, раздавая нам хорошее настроение, потерянное еще зимой под горой домашнего задания, дул легкий приятный ветерок,~---~так что парк был забит модницами, мечтавшими выложить в~инстаграмм фото своих красиво развевающихся волос.
Улыбка то~и~дело появлялась на~моем лице, превращая в~меня в~довольного кота.
Столь раннее потепление объясняли ранней Пасхой, руководствуясь простой фразой: «~На~Пасху всегда тепло!».
Незаметно для меня пробежал март, каникулы, первая неделя апреля.
Приближалось шестнадцатое апреля~---~светлый праздник для всех христиан.
Однако я~не~верующая, а~все потому что верить в~Бога стало не~модно в~наши дни, особенно среди молодежи, то~есть студентов и~школьников, представителем которых я~и~являюсь.
Одни поясняют свою позицию количеством инцестов в~Библии и~называют христианство непоследовательным.
Другие считают, что Сатана такой обаяшка, что лучше в~него верить.
Третьи хотят насолить родителям.
Четвертые не~знают просто-напросто ни~одной молитвы.
Ну~а~пятые, как я,~пытаются найти свою дорогу и~понять, а~нужен ли~им~Бог.
Но~перед грозными и~верующими родителями мы~носим крестики и~посещаем церкви, хотя бы~по~праздникам.
Таким образом, Пасха~---~просто старая, красивая и~очень вкусная традиция.
Она привносит в~весну ожидание чуда.
 
В~тот вечер мы~возвращались поздно вечером домой из~фитнес-центра.
Мама была довольная и~что-то весело намурлыкивала под нос.
Я~стояла рядом с~ней в~ожидании троллейбуса с~головой в~телефоне.
К~слову, за~неделю до~Пасхи погода ужасно испортилась.
Пришлось доставать теплую куртку и~надевать берет, потому что ветер дул промозглый и~по~утрам накрапывал дождик.
Через несколько минут подъехал троллейбус, так как на~часах, наверное, было уже около девяти, то~народу зашло мало, а~следовательно было полно свободных мест.
Мы~сели в~конце, спиной к~водителю.
Мама уставилась в~окно, а~я~достала наушники и~включила нечто мелодичное и~бессмысленное.
Вскоре мы~подъехали к~остановке, расположенной рядом с~храмом.
Что мне по-настоящему нравится в~христианстве, так это храмы.
Они яркие, все в~украшениях, с~куполами--луковицами и~так похожи на~детскую игрушку.
Это храм не~был исключением.
Несмотря на~то, что ночь пришла еще в~половине девятого, можно было понять, что при свете дня храм голубой и~резными окнами и~уютным двориком, скрытым железной витиеватой оградой.
А~на~остановке вошли они.
Группа мужчин и~женщин, верующих.
Мамина голова упала мне на~плечо, капюшон скрыл ее~лицо.
Я~решила дать ей~поспать, все равно ехать еще долго.
В~наушниках сменилась песня.
Rammstein.
Интересно, те~люди, что стояли на~задней площадке, приняли бы~меня за~сатанистку за~то, что я~слушаю? Мне было скучно, поэтому я~решила приглядеться к~верующим.


Молодых среди них не~было.
Самому младшему по~виду около пятидесяти.
У~женщин головы покрыты платками.
Они не~разговаривали между собой, зато у~каждого в~руках была довольно большая и~скорей всего тяжелая лампадка, со~свечой внутри.
Одна старушка присела на~свободное место и~зачем-то выставила руку с~лампадкой вперед.
У~старушки было очень приятное лицо, морщины (а~их~было очень и~очень много) не~старили его, а~наоборот располагали к~себе, придавая лицу доброту и~мудрость.
Лампадка же~была стара, наверное, сделана и~куплена была еще до~войны.
Она была крупнее остальных и~насыщенного черного цвета.
Но~старость в~ней проглядывала местами облупившейся и~отошедшей краской, а~также неким налетом.
Свойственным всем старым вещам.
Но~сильнее всего выдавал ее~запах.
Как только группа верующих зашла в~троллейбус, ладан почувствовали все.
Но~так как старушка сидела ближе всего к~нам, то~к~запаху из~ее~лампады примешивался еще один, явственно различимый на~фоне всех остальных.
Не~могу описать его сейчас, но~так пахнут старые книги, что покрываются плесенью на~чердаках, носки забытые под кроватью на~даче.
У~этого запаха кислый и~резкий привкус, но~иногда он~добавляет особую ноту и~определенный шарм.


Мы~вышли на~остановку раньше, так как хотели прогуляться и~подышать свежим воздухом.
Недавно прошел дождь.
Свежесть приятно ощущалась в~носу.
Фонари освещали последождевые лужи.
Днем дувший сильно, к~вечеру ветер утих, и~теперь только изредка напоминал о~себе, играясь с~всё ещё голыми ветвями деревьев.
Мы~медленно продвигались вперед, молча, не~хотели портить красоту момента, совершенно ненужными словами.
Я~сняла наушники и~слушала ночные улицы.
Изредка проезжали машины, оставляя за~собой гудение двигателей.
А~впереди нас еле передвигая ноги шла старушка из~троллейбуса.
Я~вряд ли~поняла бы, что это она, но~запах ее~лампадки был незабываем.
Вы~только представьте ту~смесь, что ловили наши носы.
Там были старость и~ладан, свежесть и~молодость, сырость и~холод.
Не~знаю откуда, но~в~тот миг появилось волшебное ощущение чистоты и~покоя.
Мы~повернули за~угол.

 --- Дочь, а~знаешь, сегодня праздник... Да, чистый четверг~---~вдруг, наклонившись ко~мне, шёпотом сказала мама.

Я~промычала, чтобы показать, что услышала, но~не~хочу отвечать.
Теперь стало ясно, почему эти люди в~троллейбусе держали в~руках лампадки со~свечами.

~
Мы~медленно шли ещё где-то минуту.
Я~втягивала холодный воздух и~старалась запомнить этот момент как можно отчетливей.
Черноту неба, тусклый свет фонаря, волосы, прилипшие к~губам, мамину куртку в~полушаге от~меня.
Но~маме, видимо, хотелось поговорить.
 
 --- Я~знаю эту старушку.
Её~сын лечится у~меня.
Она хорошая бабушка.
Добрая.
Ей~не~повезло с~сыном.
Он~пьяница.
Сначала он~долго пил, его пару раз откачивали.
Он~не~любит мать или не~уважает.
Бессовестный человек...

Тут поток слов прервался.
Я~пнула ногой камешек, лежащий на~дороге, а~мама перевела дух, чтобы продолжить.


 ---~А~потом инсульт.
Теперь он~может ходить только по~дому, держась за~стеночку рукой.
Он~так ругается на~мать, а~она сама еле ходит.
Он~с~ней обращается, как со~слугой.
Они живут в~маленькой хибаре далеко от~центра, аж~на~Лысой Горе.
А~старушка очень хорошая.
Не~повезло ей~просто.
Вот думала, сын хоть на~старости лет опорой и~поддержкой будет, но~не~вышло.
Мне всегда так тяжело с~ней разговаривать, словно я~в~чем-то виновата перед ней... А~сына ее~вылечить уже нельзя.
Такое не~лечится...
 
Мы~продолжали идти.
Я~поправила беретку, сползшую на~лоб.
Я~не~знала, как реагировать на~это.
Я~ничем не~могла помочь.
Да~и~вряд ли~кто мог бы.
Старушка уже свернула в~сторону Лысой Горы.
Но~ее~фигура, упрямо идущая вперед, с~тяжелой лампадкой в~руке, огонек, что освещал ее~силуэт, все еще стояли передо мной.
Я~видела её, с~трудом передвигающую ноги, но~все равно идущую пешком домой, где ей~не~рады.
Я~удивлялась, где ещё в~ней держится жизнь? И~как она не~побоялась одна в~ночь идти в~церковь? Лысая Гора совершенно неприятный район.
Да~зачем нужна эта вера, чтобы ради нее пускаться в~такой путь? Но~ей~она нужна.
Старушка Верит.
Для нее, я~уверена, это не~просто красивая традиция… 