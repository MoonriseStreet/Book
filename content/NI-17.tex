\chap{НИ-17} 


\lettrine[lines=3, loversize=0.1]{О}{}на скользила в~тени... Никогда еще она не~ощущала такую легкость во~всем теле, никогда еще ее~ноги не~были так быстры и~проворны, никогда еще её~сердце так сладко не~сжималось в~груди от~волнения, а~кровь так быстро не~бежала по~венам.
Впервые в~своей жизни она делала что-то настолько противозаконное и~опасное.
Нет, почти вся её~жизнь состояла из~ежедневного нарушения правил, постоянного неподчинения и~скрытого протеста, но~сейчас она нарывалась на~пожизненное заключение.
Ее~это не~пугало, наоборот придавало ускорения и~желания двигаться вперед, бежать навстречу неизвестному.

Здесь, внизу, царил серый цвет.
Сам воздух и~мелкая пыль, что причудливо кружилась в~нем, были серыми.
Если бы~не~кислородная маска, Эви не~продержалась бы~и~пары секунд.
Свет тоже был серым, но~более светлого оттенка.
Солнце уже почти сто лет не~заглядывало сюда, по~воле людей, конечно.
Здания, их~краска с~годами облупилась и~облезла неровными кусками, от~сырости давно сгнили и~тоже начали отсвечивать серым.
Провалы, образовавшиеся на~месте бывших когда-то здесь окон, и~тени, что отбрасывали здания, были насыщенного темного серого цвета, почти чёрного.
Нижние миры, как в~шутку их~прозвали обитатели Верних этажей, могли на~самом деле считаться царством теней.
Ведь тень --- то, что возникает на~границе добра и~зла, света и~тьмы, черного и~белого, а~следовательно она сера, как одежда Эви в~этот самый миг.
 
Вообще, экипировка Эви была последней военной разработкой, но~добрый папа подарил пилотный образец своему капризному чаду еще в~том году, как игрушку, про которую вскоре забудут.
Это была защитная одежда со~встроенной системой «Хамелеон», позволявшая владельцу беспрепятственно проходить мимо разных датчиков и~систем слежения.
Но~самым большим плюсом этой экипировки была обувь.
Удобные легкие берцы, имевшие небольшое родство с~кроссовками, с~вентиляцией, так что ноги не~потели после часа умеренного бега и~не~уставали, скрадывали звук шагов.

Эви подошла к~границе сектора.
То, что было до~этого, на~самом деле разминка.
Начинается самая трудная часть, но~нужно же~будет и~обратно как-то возвращаться.
Но~подумать об~этом Эви уже не~успела.
К~ее~огромной радости, ей~навстречу, издавая громкое жужжание, резко звучавшее в~тишине, и~потому создающее резонанс, несся дрон.
По~форме он~напоминал сферу, гладко-черную, с~глянцевым блеском, что тоже выделяло его на~фоне всего остального, имевшего хоть и~разные оттенки, но~одного серого цвета.
Издалека он~был похож на~маленькую черную точку.
Но~он~становился все ближе и~ближе и~уже напоминал жирную муху, которую эви видела на~картинке в~учебнике по~биологии.
Раньше мухи жили здесь и~сильно докучали местным жителям.
Но~теперь тут не~живут даже люди.
Скорость у~дрона была большая, меньше чем за~пару секунд он~достиг Эви и~тепрь завис перед ней, хотя находился до~этого на~другой границе сектора.
Значит, защита даже на~нижних мирах работает превосходно.
А~как же~иначе объяснить то, что дрон, словно хищник, почуявший добычу, так резко рванул в~ее~сторону.
А~ведь она только слегка приподняла полу защитного плаща, специально, конечно.
 
Бедный дрон, жаль он~не~может знать, как гнусно будет использован сейчас, как жестоко над ним надругаются и~используют его.
А~заодно и~всю правительственную систему.
Эви чуть не~расхохоталась в~голос.
Но~вовремя сдержалась.
А~пока, наивный дрон висел прямо перед ней в~серой густоте воздуха и~неистово громко жужжал.
Эви подняла правую руку, чуть оголила запястье, на~котором был браслет, и~поднесла его к~считывающему устройству дрона.
Мамин браслет, украденный из~косметички.
Он~открывал полный доступ ко~всей информации, лежащей в~голове дрона.
Через пару секунд, она знала, куда точно ей~следует идти.
На~картах, в~памяти дрона, это место значилось как заброшенная ТЭЦ, но~только Эви это не~интересовало.
Ей~нужно было то, что находилось немного поодаль и~намного глубже под землей.
Мало кто знал, и~слава всем тем богам, которых придумало людское воображение, но~под городом была широко разветвленная сеть катакомб, и~один из~входов в~нее был расположен в~ТЭЦ.

Эви вышла из~тени здания, чтобы тут же~нырнуть в~другую.
Она двигалась быстро, но~вдруг до~ее~ушей долетело странное жужжание.
Поблагодарив свою интуицию за~то, что сохранила управление дроном за~собой, она отправила вышеозначенного дрона к~месту, от~которого исходило жужжание.
Как ни~велико было ее~удивление, когда она увидела там людей, но~все же~наибольшее удивление вызвало то, что она не~увидела практически никакой защиты на~них.
Они дышали через кислородные баллоны, но~кислородные маски намного удобнее и~практичнее и~скрывают лицо полностью.
Эти люди были без какой-либо защиты на~одежде, и~их~руки были без перчаток, а~насколько Эви знала, уровень радиации в~этих районах все еще очень высок.
Рядом с~ними был дрон, но~не~такой как приятель Эви.
Этот издалека напоминал человека, но~вот то, что служит людям головой, было у~него сильно вытянуто вперед, и~сбоку он~походил на~букву «Г».
Второй дрон не~обратил внимания на~своего меньшего собрата и~сосредоточил свое внимание на~людях.
Это позволило Эви подлететь к~ним поближе и~рассмотреть их~лица.
Они все были мужчинами, причем сильно заросшими и~неухоженными.
А~еще они все были не~молодыми.
Глубокие морщины прорезали их~лица, сделав похожими на~кору деревьев в~парке Миллениум.
Но~Эви больше поразило то~изнеможение, сильная усталость и~глубокая печаль, что отпечатались на~их~лицах.
Среди них выделялось одно лицо.
Где-то Эви его уже видела…
 
Но~не~сейчас.
Ей~следует поспешить.
Она подумает об~этом потом.
Все равно фотографии их~лиц были загружены к~ней в~браслет.
Дрона она отпустила, предварительно подправив ему память.
Теперь он~был бы~ненужной помехой.
Хорошо, что раньше люди строили дома так близко друг к~другу.
Спасительная тень была всегда под рукой.
Так, словно став призраком, Эви добралась до~ТЭЦ.
Как и~все здания внизу, здание ТЭЦ тоже было разрушено плесенью, сыростью и~временем.
Внутри ее~встретила та~же~тишина, проникнувшая в~город вместе с~серым цветом.
Но~Эви интересовал только вход в~катакомбы.
В~инструкциях было написано, что он~в~восточной части здания, на~первом этаже, в~бывшей подсобке охраны.
Эви нашла то~место, где раньше сидел охранник.
Под слоем пыли и~грязи были погребены старые модели компьютеров, большое кресло и~несколько шкафов.
Они рассохлись, развалились, по~их~силуэтам, можно еще было угадать, для чего их~использовали.
Пройдет десять лет, хотя нет, намного меньше, и~будет уже не~угадать, что есть что.
Несколько стен обвалилось, придав помещению совершенно несолидный и~неопрятный вид.
Но~сделав пару шагов вперед, Эви обнаружила зияющую дыру -~вход в~катакомбы.
 
Фонарь был совсем новый, тоже одна из~военных разработок, не~требующая зарядки, изменяющая интенсивность свечения, от~того он~испускал холодные белые лучи с~синим отливом.
Наверное, люди, разрабатывающие военную экипировку сами очень холодные, потому что Эви от~этого света сделалось слегка неуютно.
Катакомбы встретили ее~еще более тихой тишиной, если можно так сказать, по~сравнению с~Нижними мирами.
Но~цветовая гамма здесь была совсем иная.
Если ТЭЦ принадлежало к~царству теней, то~катакомбы были царством тьмы.
Все вокруг было черным.
Не~видно было ничего, спасал только фонарь.
Казалось, что чернота густая, словно желе, и~ты~в~ней вязнешь.
Эви высоко подняла руку с~фонарем.
Сразу стало видно ,~что она в~туннеле, неизвестной длины, с~голыми стенами, покрытыми плесенью, и~земляным полом.
Ей~нужно было идти прямо все время, пока она не~наткнется на~дверь.
И~хотя прошло не~более двух минут, с~того времени, как она сюда вошла, но~ей~показалось, что она пробыла в~катакомбах несколько часов.
И~тогда Эви побежала.
Она не~слышала звука шагов, но~чувствовала как бежит, как быстро несут ее~ноги и~как сильно стучит ее~сердце…
 
Эви сбила дыхание, поэтому сейчас судорожно дышала.
Перед ней была дверь.
Совершенно новая, с~гладкой поверхностью и~особым замком.
Эви получила ключ.
Она открыла дверь и~вошла, а~затем посмотрела на~часы.
Рано.
Но~это значит, что у~неё есть несколько минут, чтобы оглядеться.
Комната была большой и~круглой.
Посередине стоял круглый стол, непостижимых размеров, заполонивший всю комнату, что навевало мысли о~короле Артуре и~рыцарях круглого стола.
Лампы наверху громко жужжали и~светили зеленым.
Снять капюшон и~кислородную маску Эви и~не~подумала.
Не~за~чем остальным видеть ее~лицо.
Она вновь бросила быстрый взгляд на~часы.
Ровно девять.
 
В~это же~мгновение в~стенах комнаты появились двери, которые сразу открылись, впустив людей.
Их~было так много, что в~комнате сразу стало тесно.
Не~десять, не~двенадцать, а~около сорока человек.
Все они были одеты так же, как и~Эви, в~плащ-хамелеон, темные берцы и~абсолютно все имели на~лице кислородные маски.
У~некоторых еще были встроены устройства, которые изменяли голос.
Тишина отступила, послышались голоса и~смех.
Кто-то обнимался, кто-то жал руки.
Несмотря на~столь сложную защиту, они все равно узнавали друг друга.
Вскоре в~стене появилась еще одна дверь.
Она медленно отъехала в~сторону, и~на~порог комнаты вступил мужчина.
Почему Эви подумала, что это мужчина? Потому что если бы~женщина была такой, то~ее~было бы~очень жалко.
Этот субъект имел рост более шести с~половиной футов и~необычайно широкие плечи.
Он~извинился за~опоздание, подошел к~столу, набрал какую-то комбинацию цифр на~приборной панели и~начал собрание.
Остальные сразу же~подошли к~столу.
Эви в~это время присматривалась к~этим людям, пытаясь определить, где тот, что позвал ее~сюда.

В~письме было сказано, что она поймет, кто ее~нанял.
Интересно только как? Поэтому решив, что наниматель сам ее~найдет, она прислонилась к~стене и~стала прислушиваться к~тому, о~чем говорили эти люди.
В~начале шел идеологический бред, свойственный каждой подпольной организации.
Потом они заговорили об~оружии, схемах доставки, объединении городов, складах.
Но~вот речь зашла об~организации безопасности доставок груза.
Кибербезопасности.
Эви навострила уши.
Она знала, что это должно напрямую касаться её~нанимателя.
Субъект, шести футового роста, оказавшийся главой этой группы организации, кивнул в~сторону человека, стоявшего слева от~него, и~вопрос отпал.
Человек, до~этого что-то рассматривавший на~столе, поднял голову и~встретился со~взглядом Эви, он~медленно ей~кивнул.
Эви пришлось обойти половину стола, чтобы добраться до~него.
Хорошо, что на~неё мало кто обращал внимание.
Как только она подошла к~нему, человек сделал шаг назад от~своего главы, повернулся к~Эви лицом и~сделал знак следовать за~ним.
В~стене снова появилась дверь.
Она вела в~еще одно помещение, меньшего размера, с~белым светом лап.
Человек сел на~большое кресло стоявшее ближе к~правому углу комнаты и~знаком велел Эви сесть на~стул, стоящий напротив него.
Через минуту молчания человек заговорил: <<Я~сейчас должен быть на~собрании.
Ждите меня здесь.>>
В~его интонациях сквозила такая уверенность в~себе, своей важности, что Эви состроила недовольное выражение лица, которое, к~счастью, скрыла маска.
И~всё.
Не~сказав больше ни~слова, человек покинул комнату, оставив Эви в~одиночестве ждать конца собрания.
Недовольно хмыкнув, она приготовилась ждать.

Внезапно взвыли сирены, предупреждающие об~опасности.
Эви, оказавщаяся в~этой комнате, как в~ловушке, начала судорожно искать выход.
Приказав, панике отступить, она решила использовать свой браслет.
Дверь поддалась.
Лампы светили с~перебоями, отбрасывая на~стены зеленые блики.
В~комнате, что странно, не~царила паника.
Глава громко, перекрывая гул сирены, сказал, чтобы все возвращались теми дверями, что пришли.
Эви попыталась протиснуться к~своей двери, благо, она помнила, где та~находится.
Но~вдруг кто-то тронул её~за~плечо.
Это оказался ее~наниматель.
Он~снова знаком приказал ей~следовать за~ним.
Неизвестно, что обуяло ее~в~этот миг.
Страх, ведь она впервые нарушала закон по~крупному, любопытство, она не~знала, кто её~наниматель, и~что конкретно ему от~нее нужно, или привычка подчиняться старшим.
Она пошла за~ним, сирена продолжала надрываться, вызывая головную боль.

Вдруг стали открываться новые двери, и~из~них начали выходить военные, в~традиционном белом обмундировании, с~парализаторами наготове.
Среди оставшихся раздались крики о~предательстве, появилась паника и~толчея.
Эви перестала понимать, что происходит.
Кто-то, неизвестно из~военных ли~или из~этих заговорщиков, не~просто достал оружие, а~применил его.
Раздались крики, потом начали стрелять уже все военные.
Синие лучи пронзали черноту, кто-то выключил свет.
Но~Эви продолжала идти за~своим нанимателем.
Вдруг он~схватил ее~за~локоть, толкнул к~двери, она ступила в~темноту.
Наниматель сзади толкнул её~и~прокричал: «Беги!».
И~она побежала, включив предварительно фонарь и~высоко подняв его над головой.
Она бежала, потому что военные вскрыли дверь и~сейчас бежали за~ними.
Наниматель откуда-то достал парализатор и~начал отстреливаться.
Ей~было страшно.
Страшно так, что хотелось остановиться и~сдаться, но~что-то внутри подстегивало бежать вперёд.
У~нее сбилось дыхание, легкие горели от~перенапряжения, холодный пот мелкими каплями катился по~спине.

Она слышала стрельбу вдали, но~уже не~могла бежать.
Ноги налились свинцом от~усталости.
Что-то холодное прикоснулось к~её~голове, и~она провалилась в~темноту…

\vspace{5mm}

Люди всегда стремились к~небу.
Часто это объяснялось желанием быть поближе к~Богу.
Но~если допустить, что Бога нет, чем же~все это тогда обосновать?
Стремлением к~совершенству?
Небо было и~остается загадкой номер один.
Что скрыто там от~нас? 
Есть ли~там ответы на~наши вопросы? 
Но~чем ближе мы~по~нашему мнению подходим к~небу, тем сильнее мы~на~самом деле отдаляемся от~него.
Не~этой ли~причиной можно объяснить то, что люди перестали селиться на~земле.
Они начали строить здания сначала по~сто этажей, и~они казались им~необыкновенно высокими, потом количество этажей с~завидным постоянством стало расти.
Таким образом, люди строили дома все выше и~выше и~даже не~заметили, когда начали жить за~облаками.
Конечно, это создавало неудобства для самолетов, но~эта проблема была быстро решена.
Фундамент, как и~раньше находился на~земле, но~на~первых четырехстах этажах никто никогда не~жил, дальше шли так называемые промышленные этажи, на~которых находились разнообразные предприятия разной степени вредности.
И~только после них на~семисотых или восьмисотых этажах начинались жилые помещения.
На~самом деле, это было очень удобно.
Популяция человечества в~первые годы сильно увеличивалась, поэтому широкие здания по~тысяче, а~то~и~полторы тысячи этажей прекрасно подходили для нужд городов.
Стиль зданий был современным.
Много стекла, хрома, больше открытых пространств.
В~моду вошел минимализм.
Да, людям и~не~нужно было множество вещей.
Только компьютерная комната...

Солнечные лучи, не~встретив препятствий, скользнули в~спальню.
Огромные окна в~пол им~это позволяли.
Раньше, люди всегда могли сказать, какое время года, не~ориентируясь по~солнечным лучам.
Теперь же, только время их~появления могло сказать какой сейчас сезон.
Но~людям и~не~нужно и~не~интересно было знать, зима ли~сейчас или весна.
Хотя Эви точно знала, что зима.
Если бы~люди продолжали жить как раньше, то~многие выйдя на~работу бы~сегодня, втянули бы~носами бодрящий зимний воздух и~заспешили бы~по~делам, получив свою долю свежести.
К~сожалению, выйти на~улицу уже не~получится.
Выходить особо некуда.
Да~и~здесь, наверху, воздух другой, разреженный.
Было девять утра.
Лучи нагло пробежали по~холодному кафелю, поднялись по~одеялу и~добрались до~лица Эви.
Она неохотно приоткрыла один глаз, чтобы тут же~его зажмурить.
Но~дело было уже сделано --- Эви проснулась.
Она сбросила одеяло и~вяло пошла на~кухню, чтобы сварить кофе.
Родители, видно, так и~не~вернулись домой с~ночного совещания.
На~столе стояли немытые чашки тарелки, а~также неровной горкой были навалены вчерашние газеты.
Отец признавал только газеты.
Они стоили, конечно, баснословных денег, но~благо её~семья могла себе их~позволить.
Правда, что-то на~столе в~это утро было не~так.
Среди обычного бардака, создающего атмосферу жилого помещения, скрывалось нечто чуждое.
Сосредоточившись, Эви увидела ослепительно-белую бумагу конверта.
Такой же~мерзкий цвет был и~у~стен проходов и~общественных комнат.
Герб клиники.
На~все города это была единственная клиника.
Холодок ужаса пробрал Эви.

Звонок в~дверь.
Быстрый взгляд на~домашнего дрона, парящего невдалеке.
На~его панели высветилось изображение.
Лицо друга.
Губы непроизвольно улыбнулись, и~страх ненадолго отступил.
Эви кивнула дрону, стены разъехались, и~в~комнате оказался юноша, ровесник Эви.
А~также однокурсник и~лучший друг, который старался её~не~оставлять.
Тот~раскинул руки, и~Эви, не~задумываясь, приняла его объятия.
Было тепло и~хорошо.
Уютно.
Но~это~ощущение быстро рассеялось.
Друг схватил ее~правой рукой за~подбородок, приподнял голову так, чтобы видеть ее~глаза.
И~огонёк понимания проскользнул в~его взгляде.
Затем взгляд начал темнеть, а~на~скуле угрожающе заиграл желвак.
Миг, после которого звук пощечины прорезал воздух.
Боль накрыла через несколько секунд.

Глаза резало и~щипало.
Джеймс жестоко ухмылялся, заливая эту дрянь ей~в~глаза.
Он~был прав.
Хотя это и~ужасно больно, но~ничего другого, что скрыло бы~опасную красноту, люди придумать так и~не~смогли.
Джеймс заставил ее~принять душ, плотно позавтракать.
Потом он~подобрал ей~наряд, который не~выделял бы~ее~из~толпы.
Стены разъехались вновь.
Белый свет, белый коридор, белые плащи.
Только их~лица вмешивались в~эту гамму.
Общие комнаты находились на~самых верхних этажах, поэтому им~предстояло подниматься на~лифте.
Бесконечная анфилада залов с~потолками высотой до~двадцати футов, которых казалось и~не~было вовсе.
Все здесь было из~стекла --- стены, потолки.
Но~все равно хотелось верить, что и~в~самом деле среди облаков.
Среди этого царства грез, подсвеченного солнечными лучами, казавшегося невероятно мягким и~прекрасным, что хотелось до~него дотронуться, но~в~то~же~время появлялась боязнь неосторожным движением разрушить эту нечеловеческую красоту.
От~погружения в~мечты и~умиление спасали лишь кое-где видневшиеся серебристые опоры, да~знание законов физики и~здравого смысла.
Но~Джеймсу явно было не~созерцания облаков.
Крепко держа Эви за~руку, он~упорно двигался вперед, стараясь не~слишком задерживаться за~разговорами с~многочисленными друзьями.
Фил жил в~другом здании, там же, где находились учебные комнаты и~до~которых было два прохода.
К~счастью, большая часть студентов уже была на~занятиях, и~на~проходах не~создавалось обычных для этой части утра человеческих заторов.
Охране они сказали, что проспали лекции и~сейчас хотят наверстать упущенное.
Конечно, можно и~прямую трансляцию посмотреть и~записи потом послушать.
Но~на~профессоров приятней смотреть вживую.
 
Фил встретил их~радушно.
Сложно в~этом мире было сказать, когда человек рад тебе по-настоящему, а~когда соблюдает вежливость и~осторожность.
Люди хорошо научились прятать свои настоящие эмоции.
Хотя официально говорилось, что полиция мысли выдумки дураков, которым нужна только анархия, но~все-то знали, что она существует.
Да~и~офицеры полиции мысли не~особо таились.
Они могли даже днем забрать человека.
Но~камеры всегда можно обмануть, если знать нужных людей.
Для начала Джеймс и~Фил проверили вчерашние запаси, пока Эви вела себя, как подобает.
Но~вчера вечером она перешла через три прохода в~черном плаще и~спустилась на~лифте на~сотый этаж.
Потом она пропала с~камер, а~возвращалась уже в~три ночи.
И~что самое ужасное, камеры зафиксировали красные глаза.
Фил обернулся к~ней, сидя в~кресле, и~погрозил пальцем, сказав: <<Девушка, тебе повезло, что я~вчера тебе отследил и~вывел эти данные.
Я~еще вчера за~тобой все подчистил.
Сколько раз предупреждал: НИ-17 принимай лучше у~меня, тут я~точно за~тобой все приберу>>.
Джеймс зло скрипнул зубами и~пробомотал, что лучше бы~она эту дрянь вообще не~принимала.

Джеймс сказал, что лучше пару часов посидеть в~кафе или погулять по~Зимнему саду, чтобы на~всякий случай запутать камеры.
Таким образом они оказались в~этом здании на~окраине города в~самом, из~ныне построенных, высоком здании.
Две тысячи этажей.
На~самом верху находлся зимний сад, но~эти цветы, столь прекрасные, высокие деревья и~трава нужны были только для создания кислорода, а~их~внешний вид мало заботил правительство.
В~итоге это было странное нелепое нагромождение чего бы~то~ни~было и~своей нелепостью оно часто отпугивало Эви.
Поэтому Джеймс отвел ее~в~зону кафетериев.
Онa пила апельсиновый сок, который на~самом деле был простой водой.
Но~имплантат в~ее~голове заставлял ее~видеть оранжевую жидкость в~бокале, нос чувствовать запах апельсинов, но~только на~вкус это была самая обычная вода без вкуса, без запаха, без цвета.
Многие даже не~знали, что они пьют на~самом деле.
Эви довелось однажды попробовать настоящий апельсиновый сок, настоящий кофе и~настоящий чай.
Да, по~утрам на~самом деле она и~все люди городов вместо кофе или чая пьет очень горячую воду.
Только тот, кто пробовал настоящее, может сказать, что правительство их~обманывает каждый день, но~кто поверит этой горстке счастливчиков, познавших настоящий вкус жизни? 

Эви решила осмотреться.
Она сидела среди таких же~обманутых, как и~она людей.
Но~те, счастливо потягивали обман из~чашек, стаканов, кружек.
Кто знает, может они пьют на~самом деле из~листьев, а~имплантат заставляет воспринимать окружающую действительность по-другому? Она уже ничего не~знает и~ничего не~понимает.
Все, что она помнит, это боль холодное прикосновение к~затылку чего-то холодного и~темнота, в~которую она летит.
А~потом воспоминания хлынули в~нее, словно лавина.
Усмешка, больше похожая на~гримасу боли и~отчаяния искривила ее~лицо.
Хорошо, что капюшон скрыл это от~вездесущих камер.
Теперь она по-другому взглянула на~сидящих рядом с~ней.
Ох, до~чего же~они хилые, по~сравнению с~теми, кто окружал ее~вчера.
Или позавчера? После НИ-17 целый день почти ничего не~помнишь, и~часто бываешь сильно дезориентирован.
Эти люди вокруг все как на~подбор высокие, худые, сутулые, с~впалыми щеками, рахитичными фигурами, серым, даже каким-то землистым цветом лица.
Все одинаково похожие друг на~друга.
В~глазах скрытое безумие, а~плоть --- отражение беспорядка, царящего в~их~душах.
У~них длинные паукообразные пальцы, холодные как лед.
Их~касания неприятны и~заставляют морщиться от~отвращения.
Волосы засалены и~немыты.
Но~это не~от~отсутствия шампуней или денег.
Не~от~лени,~а~от~нежелания следить за~собой.
Да~и~зачем? Эви видела как имплантаты сияют сине-изумрудными огнями под их~кожей.
Они все не~здесь.
Сейчас они подсоединены к~Сети.
Поэтому они с~таким удовольствием глотают безвкусную воду.
Там, в~мире грез можно быть кем угодно, и~как угодно выглядеть.
Там они по-настоящему свободны.
И~одежда на~них серая, такого же~цвета, как их~жизнь вне Сети.
И~как она раньше этого не~понимала? Нет, не~нужно себя обманывать.
Она всегда знала об~этом, просто сегодня дала волю разрозненным образам собраться в~единый поток связных мыслей.

Она поймала взгляд Джеймса на~себе.
Тот послал по~защищенному каналу Сети сигнал: <<Вижу, тебе уже лучше?>>
Она кивнула, затем снова поймала его сообщение: <<Идем домой?>>.
И~снова кивок.
Почему-то горло не~хотело генерировать звуки.
А~те~в~свою очередь не~хотели вылетать наружу.
Он~подошел, взял ее~за~руку, и~они молча двинулись домой.

Стены раздвинулись, впустив их~в~квартиру Эви.
Джеймс заметил, что её~лицо снова приобрело то~выражение, что и~всегда.
Холодное, враждебное, циничное, надменное.
Она снова надела свою защитную маску.
И~дальше он~не~смог уже сдерживаться, как сдерживал эти порывы все эти годы.
 
 --- Прошу, не~губи себя.
Не~нужно, Эви. --- в~ответ ему только тишина, да~еще более надменное выражение лица.

Он~перевел дыхание, чтобы не~взорваться совсем.
 
 --- Я~хочу знать.
Почему? Я~имею на~это право больше, чем кто-либо еще.

Она колебалась.
Потом скинула плащ, туфли.
Босыми ногами дошла до~секретера отца.
Набрал код, достала портсигар.
Невесело ухмыльнувшись, чуть пошатываясь, как если бы~была пьяна, дошла до~противоположного конца комнаты, попутно зажигая длинную сигару.
Он~не~знал, откуда, но~ее~отец всегда держал дома сигары.
Она знала, что Джеймс ненавидит, когда она курит.
Но~сегодня все делала, чтобы его позлить.
 
 --- Знаешь, мой друг, а~мне прислали письмо.
Из~клиники.
 
Джеймс облегченно перевел дух.
Наконец-то родители решили заняться жизнью своей дочери.
 
 --- Я~знал, что рано или поздно это случится.
Понимаю, тебе это не~по~душе.
Но~мне кажется, что для тебя это будет лучше.
Не~молчи! --- он снова не~удержал себя.
 
 --- Возможно... --- еле слышно, с~какой-то издевкой в~голосе произнесла она.
 
 --- От НИ-17 невозможно вылечиться. Мне там будут давать его столько, что вскоре я~умру от~передозировки... --- смеясь, через секунду продолжила она, но~договорить не~успела.
 
Джеймс стремительно подошел к~ней сзади, что она и~не~заметила, а~может не~хотела замечать.
Он~вырвал сигару из~ее~пальцев и~зло бросил:

 --- Ты~можешь отталкивать меня столько, сколько сочтешь нужным.
Но~я~все равно добьюсь правды.
Я~её~заслужил.
А~еще я~заслуживаю твоего ко~мне уважения.
Мы~поговорим, когда ты~придешь в~себя.
Я~попробую сделать так, чтобы ни~одна порция этой дряни до~тебя не~дошла.

И~он~ушел, растаяв в~стене.
Она подумала, что если бы~в~её~квартире были настоящие двери, как у~людей, что жили в~домах, максимум в~сто этажей, то~он~бы~непременно хлопнул этой дверью.
Она прислонилась к~окну.
Ей~хотелось, чтобы пошел настоящий ливень, чтобы окружающий пейзаж отражал состояние ее~души.
Но~за~окном были облака, весело подсвеченные солнечными лучами.
Слеза скатилась по~стеклу.
Её~можно было принять за~дождевую каплю.
 
Эви утерла слезы и~покачивающейся походкой зашла в~отцовский кабинет.
Там были её~дозы НИ-17, заботливо приготовленные маминой рукой, а~также еще одна редкость --- настоящая бумага.
Эви достала шариковую ручку и~начала писать.
 
 \begin{center}
 	***
 \end{center}
{\tt
\setlength{\leftskip}{1em}
\setlength{\rightskip}{6em}
<<В~конце двадцать первого века были впервые изобретены импланты, что позволило людям перенести сознание в~виртуальную реальность.
Теперь их~жизнь изменилась.
Многие в~то~время думали, что в~лучшую сторону.
Да.
Войны прекратились в~реальном мире.
Люди перестали отравлять атмосферу своим отвратительным оружием.
Но~это не~означало, что войны закончились на~самом деле.
Войны перешли на~новый этап.
Это была война в~виртуальной реальности.
Люди настолько ушли в~виртуальный мир, что позабыли о~других науках.
Тогда ученые, впервые предоставленные самим себе начали восстанавливать то, что было так неосторожно уничтожено ими же~самими.
Они поняли, что найти ответы на~многие вопросы, что всегда волновали умы людей, можно лишь воссоздав приблизительно ту~среду, которая была, когда люди только появились.
Ученые пошли на~сделку с~правительством.
Для простых людей строят огромные дома, чтобы не~мешаться господам ученым, а~те~возвращают чистый воздух и~исчезнувших животных, попутно занимаются исследованиями космоса, думают как вылететь человеку за~пределы атмосферу.
в~общем занимаются Наукой, а~не~придумывают новые способы человекоубийства.
За~это ученые создали кое-что, что позволило правительству контролировать мысли простых людей.
Дали им~абсолютную власть над простыми смертными.
А~так как же~постоянно подсовывают новые военные разработки, биологическое оружие, новые виды камер и~дронов.
Мои родители --- оба ученые.
Мама ---врач, вернее она что-то вроде психолога.
У~нее исследование.
Жизнь простых смертных.
Она всех изучает и~делает какие-то выводы.
Меня она, кстати, тоже изучает.Отец конструирует дронов.
Они как бы~ученые, но~в~то~же~время и~на~правительство работают.
Поэтому за~мной почти никогда не~следят камеры.
Поэтому у~меня браслет, открывающий доступ почти ко~всему.
Я~часто была снаружи, за~пределами городов.
Я~знаю, как там хорошо сейчас.
Постоянно проводятся тесты среди детей на~нахождение склонности к~каким-либо наукам.
Моим родителям не~повезло.
Их~единственная дочь годится только в~искусствоведы.
Мама в~тот день чуть не~умерла от~горя.
Потом я~всю жизнь, живя с~нею чувствовала свою ущербность.
Зато отец меня любит.
Я~люблю с~ним сидеть в~настоящем лесу.
Но~только со~смертью родителей я~не~смогу больше довольствоваться всем этим.
Поэтому у~меня с~ними договор.

Когда я~училась в~школе, то~познакомилась с~парой девочек.
Они были глуповатыми и~погруженными в~мир грёз, но~они были по-настоящему добрыми.
Они всегда мне поднимали настроение, поддерживали.
И~все эти люди, что ушли в~мир грёз.
Среди них не~было ни~плохих, ни~хороших.
Они были обычными, радовались жизни, солнцу, весне.
Так по~какому праву это у~них отобрали? Не~они загрязняли биосферу, не~они вели разрушительные войны.
Пусть без ученых и~правителей они и~остались жить в~каменном веке, зато они были бы~по-своему счастливы.
Они умеют находить луч солнца даже во~тьме.
Тогда как я~и~прочие, выросшие в~моей среде, не~видят удовлетворения.
Они не~могут остановиться.
На~самом деле снаружи не~все так радужно.
Восстановить утраченное не~получилось.
Большая часть Земли --- сухая пустыня.
Человеческий ген стареет, идеи ученых тоже измельчали.
Люди почти бесплодны, земли тоже.
Мы~пошли не~по~тому пути.
Все благие начинания обернулись крахом.
И~хотя ученые и~правители мнят себя выше остальных, на~самом деле они так же~слепы.
Только слепота вторых оправдывается тем, что их~сделали слепыми, а~первые так и~не~смогли выйти из~своих мелких проблем.
Чем сильнее мы~развиваем новые технологии, тем дальше мы~от~звезд.
Мы~не~ищем путей спасения, мы~сами роем себе могилу.
Но~может там, вдалеке, кроется спасение?

Как ты~понимаешь, когда такое выливается на~тебя в~шестнадцать лет, шок будет незабываемым.
Я~пыталась бороться.
Агитация, зубрила физику.
Читала про Циолковского, Королева.
Ничего.
Был выход.
НИ-17 --- новый вид наркотика.
Знаешь, у~нас наркотики запрещены, а~употребление карается пожизненным заключением и~стиранием памяти.
Это потому что в~нас самих каждый день имплантатом создается какой-то вид наркотика, чтобы правительство могло нас контролировать.
А~наркотик плюс наркотик получается разлад, и~человек уже видит реальность такой, какая она есть.
Но~с~помощью НИ-17 можно создать в~своем воображении свою собственную виртуальную реальность.
В~своей я~борюсь со~злом и~являюсь революционером.
Через год от~частого употребления НИ-17 я~умру.
Для меня это наилучший выход.
Я~собираюсь исчезнуть, пожить на~Нижних мирах.
Кстати, те, кто употреблял героин или кокаин тоже живут там, только их~заставляют без защиты выполнять какую-то грязную работенку.
Мне там самое место.

Не~ищи меня.

Я~люблю тебя. Прощай.

Теперь ты~знаешь все, Джеймс.

\begin{flushright}
Твоя Эви.>>
\end{flushright}
}
\begin{center}
	***
\end{center}
 
Дописав письмо, она сложила его в~конверт и~поставила печать, как будто жила в~таинственном восемнадцатом веке.
Она взяла НИ-17 со~стола, вышла из~кабинета и~подошла к~окну.
Бросила печальный взгляд на~облака.
Потом одела черный плащ, с~защитой «Хамелеон».
Обула берцы.
Кислородную маску пока просто взяла в~руки.
Стена отъехала, показав огромную комнату с~круглым столом посередине и~зеленым освещением.
Эви с~улыбкой шагнула туда.
Шприц, через который вводился НИ-17 был пуст.

