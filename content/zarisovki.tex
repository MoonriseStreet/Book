\chaps{***}
\lettrine[lines=3, loversize=0.1]{К}{}ажется, в~земной литературе был такой жанр, как романтическая поэма.
Но~самое интересное заключалось в~том, что композиция поэмы называлась вершинной.
Она открывала читателям только самые значимые события в~жизни героя.
Если бы~мы~могли нанизать нашу жизнь на~нитку, она бы~вся там не~поместилась.
В~любом случае пришлось бы~удалять ненужные события, незначительные элементы.
Мы~бы~снова оказались перед выбором.
Что бы~мы~оставили? Готова побиться об~заклад, что больше всего там было бы~глупых, но~красочных моментов, во~время которых мы~чувствовали себя счастливыми.
Потом добавились бы~пара тройка тайн, капля истерик, воспоминание о~первой любви, события, заставившиеся нас задуматься, переосмыслить себя (конечно, если таковые были в~нашей жизни).
Но~были бы~нанизаны на~нить нашей жизни улыбки прохожих, лица первых встречных, ступеньки в~подъезде, листья деревьев, каждодневный завтрак? Да~и~зачем? Это ведь \textit{наша} жизнь.
Мы~вспомним только то, что было важно для нас.
А~все остальное сотрётся со~временем, будто этого и~не~было вовсе.
А~какой смысл тогда в~этом? Почему все это случается каждый день? Эти прохожие, их~улыбки и~слезы? Зачем все это происходит, если забудется через день или час?
\clearpage
\chaps{***} 
\lettrine[lines=3, loversize=0.1]{К}{}огда мне было двенадцать лет, я~неожиданно поняла, что странно охладела к~миру, к~людям, к~событиям.
Например за~полгода до~того как меня озарило понимание, я~приехала в~летний лагерь.
Знаете, обычный летний лагерь с~полной антисанитарией, плохой едой и~курящими втихомолку подростками.
Я~поехала с~друзьями.
Они у~меня люди очень активные и~яркие.
Им~нужны приключения и~развлечения, постоянное движение.
Иногда мне кажется, что они сумасшедшие, но~я~упорно стараюсь этого не~замечать.
Не~знаю как, но~я~все же~поехала в~этот лагерь.
Говорю сразу в~первые дни мечтала просто оттуда сбежать.
Общая еда, одежда, косметика, обувь, сигареты, мыло и~шампуни, ввели меня в~ступор.
Но~была во~всем этом безобразии одна вещь, из-за которой я~решилась остаться.
Танцевальный конкурс.
Я~с~детства обожаю танцы.
Как-то раз еще в~детстве на~рынке звучала музыка и~я~начала танцевать прямо там посреди улицы.
Мама говорила, что прохожие стали собираться вокруг меня, многие искали шляпу, куда можно было бы~бросить деньги.
Но~я~не~была попрошайкой, мне не~нужны были их~деньги, я~просто танцевала.
Для себя.
После этого меня отдали на~танцы, я~увлеклась ими всерьез, пока однажды делая стойку не~сломала руку.
Семейный совет назвал танцы опасными и~я~перестала даже думать о~них.
Но~там в~лагере у~меня появилась возможность снова испытать эту свободу.
В~танце есть только ты, музыка и~движение.
Ничего больше: ни~дурных мыслей, ни~глупых сомнений.
К~этому конкурсу я~готовилась две недели, упорно тренировалась, купалась в~этой атмосфере.
Потом наступил сам конкурс.
Я~не~выиграла.
Я~знала, что не~выиграю, я~слишком давно не~танцевала, да~и~не~ради места все это затевала.
Но~меня поразило больше, что когда я~танцевала в~тот день я~перестала чувствовать.
У~меня не~было эмоций.
То, что раньше рождало во~мне бурю эмоций, теперь не~вызывало ничего.
И~когда, мне вручали бумажку, с~надписью о~моем втором месте, мне уже было все равно.
И~уже через несколько недель, когда мама спросила: <<Ты~не~злишься на~нас за то, что мы~заставили тебя отказаться от~танцев?>>.
Я~честно ответила: <<Нет.
Мне уже всё равно.>>
С~этого дня я~перестала испытывать чувства, если это можно так назвать.
Я~не~стала холодной, нет, все было намного хуже, я~стала безучастной ко~всему.
Меня перестали узнавать друзья.
Да~и~мне стало с~ними тяжело.
Я~находила глупыми и~бессмысленными все их~поступки, слова... Я~перестала замечать, что вокруг меня вырастает стена отчуждения.
Я~постепенно становилась тенью, будто \textit{меня} уже не~было.
Отрезвила меня фраза, случайно увиденная в~какой-то книге: <<Когда человек утрачивает чувства, он~становится зверем, потому что вместе с~чувствами утрачивает и~душу.
Он~становится равнодушным.
А~кто может быть хуже равнодушных людей, способных пройти мимо чужого горя?>>
Я~испугалась.
Неужели я~такая? Неужели я~стала хуже зверя? В~тот миг я~твердо решила собирать моменты, которые заставили меня что-то почувствовать.
И~этот первый безотчетный страх был первым в~моей коллекции.
Потом туда добавились восхищение перед красотой неба, чувство теплоты и~защищенности маминых объятий, нежность солнечных лучей в~полдень на~паркете и~многое другое.
Со~временем коллекция чувств стала коллекцией красивых моментов.
Розовый закат на~море, цветение вишни, волосы на~ветру, тюльпаны во~время майского ливня... Но~в~двенадцать лет я~увидела подлинную утонченность.
Была зима, причем зима с~большой бувкы.
За~два дня навалило снега по~колено, а~потом в~течение недели светило не~греющее солнце, и~снег нещадно слепил глаза.
Я~шла через парк.
Вечером, часов в~пять.
Солнце уже собиралось прятаться.
Оранжевое небо с~розовыми облаками, поверх которых встают ветви деревьев в~инее, само по~себе отличное видение, но~было там еще нечто.
На~снегу, прямо на~дороге лежал не~до~конца раскрытый бутон кустовой розы нежно-бежевого цвета.
По~краям цветок был уже коричневым-верный признак увядания, но~еще не~утратил своей прелести.
Он~лежал один на~девственно белом, сверкающем, словно бриллиант снегу.
Один, уже умирающий, но~все еще прекрасный.
Я~подняла его к~небу.
Он~встречал свой Закат.
Не~могу передать тот вихрь эмоций, что захватил меня.
Если бы~я~была поэтом, я~бы~смогла сочинить прекрасное стихотворение.
Если бы~я~была японским философом, то~сочинила бы~хокку.
Если бы~я~была художником, то~нарисовала бы~целую картину, но~я~никто.
Мне ничего не~оставалось, кроме как любоваться этим моментом, впитывать каждую его деталь и~надеяться, что он~никогда не~сотрется из~моей памяти.
Я~была одновременно и~зла на~людей, посмевших сорвать еще не~раскрытый бутон, и~благодарна им~за~то, что позволили увидеть этот миг...

\clearpage
\chaps{***} 
\lettrine[lines=3, loversize=0.1]{Я}{} терпеть не~могу общественный транспорт.
Эту пытку мог придумать только извращенный человеческий мозг, оправдывая ее~создание облегчением человеческой жизни.
Особенно я~не~люблю автобусы.
Метро я~еще терплю, только там можно до~конца прочувствовать этот особый аромат спешки и~невнимательности друг к~другу, свойственный всем городам.
А~автобусы? Жуткое творение, убивающее всякую любовь к~людям.
Но~есть у~него одно замечательное свойство, оно выводит на~свет все наши пороки.
Но~сейчас мы~не~об~этом.
Однажды мне не~посчастливилось добираться до~дома именно на~автобусе.
Конечно, можно было бы~идти и~пешком.
Но~зимой, по~льду, пешие прогулки не~доставляют особого удовольствия.
Еще одна причина, почему я~так не~люблю автобусы: как мужчине, мне всегда приходится стоять.
Я~стоял, и~в~который раз проклинал московские пробки, как вдруг на~остановке (Наконец! Мы~до~нее добрались!) в~автобус вошла девушка.
От~нечего делать я~принялся её~рассматривать.
Простое пальто, джинсы, черная шапка, из-под которой выглядывают каштановые волосы.
Обычное русское лицо, но~только от~чего-то очень грустные глаза.
Она села позади.
Перед самым выходом я~обернулся, чтобы посмотреть на~неё.
Мне не~давали покоя эти грустные глаза.
По~её~лицу текли слёзы.
Но~она словно и~не~замечала их, она была погружена в~себя, в~свои чувства.
По~дороге домой я~все никак не~мог забыть её.
Я~не~понимал, какие чувства поглотили её~настолько, что она даже не~замечала слёз.
Была ли~это тоска, печаль, грусть, боль, обида, облегчение? А~ведь она еще маленькая.
Я~бы~удивился, если бы~ей~было двадцать.
Я~сам последний раз плакал ещё в~детстве, да~и~то~от~глупой обиды на~родителей.
Да что же~такое происходит с~нашим миром, что столь юные создания начинают плакать?
На~миг мне захотелось стать её~героем, совсем как в~детстве.
Я~хотел защитить её~от~слез.
Потом сообразил, что защищать нужно не~от~слез, а~от~того, что их~вызывает.
После возмутился, тем, что рядом не~было никого из~близких ей~людей, которые бы~просто смогли её~обнять, поддержать одним словом.
Но~некоторым людям становится ведь легче, поплакав в~одиночестве.
Слёзы~---~это выражение чувств.
А~может она переживала совсем по~глупой причине? Сломала ноготь, каблук, не~сдала зачёт, а~я~тут ищу для неё красивые чувства, оправдания, в~которых она не~нуждается.
Но~что-то в~её~образе меня остановило.
Она была столь сосредоточена, поглощена чем-то.
\textit{Так}~не~могут плакать из-за глупости.
Хотя, не~знаю... Жаль, что мы~никогда не~сможем до~конца понять другого человека, даже если он~нам очень близок.
 
\clearpage
\chaps{***} 
\lettrine[lines=3, loversize=0.1]{Г}{}оворят, старость наступает тогда, когда стареет душа.
Эта фраза преследовала меня всю мою жизнь.
И~сейчас я~могу с~уверенностью сказать: <<Не~верьте!>>
Чушь!
Глупость!
Че.~
Пу.~
Ха.
Когда ты~стал слаб настолько, что любое движение доставляет тебе боль, когда ты~уже не~можешь сесть на~велосипед, потому что твои кости истончились, когда не~можешь выходить на~пробежки по~утрам, а~всё, что тебе доступно, это степенный кружочек по~парку, после которого у~тебя появляется одышка, и~врачи запрещают тебе крепкий кофе, а~зрение слабо настолько, что ты~не~можешь без очков прочесть ни~строчки из~любимой книги, то~никакая молодость духа вас не~спасёт.
Уж~поверьте моему опыту.
Каким бы~молодым в~душе ты~не~был, но~если время забрало твою телесную молодость, то~всё.
Тушите свет и~ложитесь умирать с~чистой совестью! Остается лишь каждый раз смотреть в~отражение на~зеркале и~каждый раз видеть там уже другого человека, который не~живёт, а~доживает.
Как жаль, что у~нас запрещена эвтаназия.
Почему? Для людей вроде меня это наилучший способ остановить мучения.
Да, я~слабая.
Я~боюсь старости, но~не~боюсь смерти.
Смерть принесет облегчение.
Я~не~верю ни~в~Рай, ни~в~Ад, ни~уж~тем более в~Бога.
Я~всю жизнь положила на~алтарь науки, так пусть теперь наука обеспечит мне безболезненный уход.
У~меня нет семьи, детей, внуков.
Я~одна.
Родители умерли ещё давно.
Обзаводиться семьей я~не~хотела в~то~время, а~потом стало слишком поздно.
Мои студенты имеют своих родителей и~свои семьи, чтобы заботиться еще и~обо мне.
Так зачем мне длить свои мучения? Из-за глупой боязни и~гордости.
Кому я~нужна? Сама покончить жизнь самоубийством я~не~хочу, да~и~не~смогу.
Вены режут и~выбрасываются из~окон молодые, пить снотворное не~надёжно, мне нужен яд, хороший, сильный.
Но~где я~его возьму?
Да~и~что я~напишу в~своей предсмертной записке?
В~своей смерти я~никого не~виню.
Умираю сейчас, потому что не~хочу умирать от~старости.
Смешно, не~правда ли? За~всеми этими размышлениями я~вышла на~улицу.
Ужас, сколько я~шла! Раньше мне хватило бы~и~десяти секунд.
А~сейчас, эх... А~вокруг меня был чудесный зимний полдень, вернее вечер.
Солнечные лучи уже окрасили всё в~оранжевый.
На~улице стоял крепкий мороз.
Одно меня радует.
Я~живу рядом с~парком и~могу гулять в~свое удовольствие даже сейчас.
Я~медленно брела по~дорожкам, сравнивая себя Бабой Ягой из~вступления к~Руслану и~Людмиле.
Меня обгоняли молодые парочки, подростки, студенты, мамаши.
Я~нашла для себя скамеечку и~присела, чтобы перевести дух и~справиться с~одышкой.
За~скамейкой я~заметила замерзшую лужу.
Малышня, видно, уже потопталась по~ней.
Не~знаю, с~чего вдруг, я~встала, подошла к~луже и~осторожно, чтобы не~поскользнуться и~не~получить перелом шейки бедра, ступила на~лёд.
Потом, поддавшись какому-то глупому порыву, топнула ногой.
Я~повторила это ещё раз, только не~слегка, как в~первый раз, а~вложив всю силу.
С~наслаждением услышала хруст льда, а~потом расхохоталась громко, сильно, как в~детстве.
Повезло, что меня никто не~заметил, потому что если бы~заметили, то~непременно отправили бы~в~психбольницу.
Но~мне было хорошо, как в~детстве, и~очень тепло, и~уютно.
Не~было больше отчаяния.
Наверное, я~впала в~детство, как все старики.
Не~знаю, хорошо это или плохо, но~если в~моей старости будет больше таких моментов, то~может мне и~удастся спокойно дождаться смерти, не~сильно мучаясь?
\clearpage
\chaps{***} 
\lettrine[lines=3, loversize=0.1]{М}{}ы~никогда не~обращаем внимания на~людей, которые выполняют мотонную работу.
Да,да вы~угадали, я~сейчас говорю о~фабричных рабочих.
Они каждый день делают одно и~то~же, одно и~тоже.
Из~года в~год, пока сами не~становятся как фабрика, на~которой они работают.
Мы~презрительно кривим носы, услышав об~этих людях и~их~трудностях, но, тем не~менее, они тоже люди.
И~они тоже, как и~мы~ходят в~те~же~магазины, смотрят те~же~телепрограммы и~фильмы, читают те~же~книги.
Нередко среди этих людей можно найти удивительных философов, ведь однообразный труд способствует раздумью.
Им~ничто не~мешает анализировать, мечтать, ведь нужно же~как-то заполнять мозг в~то~время, как руки делают что-то совершенно автоматически.
Мы~не~любим таких людей,ведь у~них была возможность изменить свою жизнь, но~они упустили свой шанс.
Стоило прилежнее учиться в~школе, настоять на~своем в~каком-то споре и~жизнь была бы~другой... Не~правда ли? Фабрика сменилась бы~офисом, и~всё осталось бы~также.
Человек так и~не~поднял бы~головы и~не~посмотрел на~небо.
Он~бы~снова погряз в~своих проблемах.
Но~был бы~он~несчастен? Мы~не~любим таких людей, потому что они счастливы и~без высшей цели в~жизни, они не~мучаются вопросами о~смысле жизни и~своей цели, они не~ищут ответов.
Может мы~просто завидуем им? Их~спокойствию, их~счастью.
Может от~того мы~ищем цель, потому что не~смогли найти свое счастье? А~были ли~мы~созданы, чтобы просто быть счастливыми? Не~слишком ли~это просто? Я~не~верю в~это.
Я~знаю, что весь наш монотонный труд, каждодневные проблемы~---~все это нужно, чтобы заполнить пробелы, создать почву для того, чтобы родилось что-то великое.
Когда-нибудь я~надеюсь, я~увижу результаты всего этого.
Но~пока единственное оправдание, которое я~могу найти для существования прохожих, ежедневных ритуалов и~даже написанных мною слов~---~вера в~то, что мы~нужны для чего-то большего, нежели тихое счастье.
Когда-нибудь это станет нужным.
Всё это.
Кому-нибудь.
Для всего в~этом мире есть цель.
По~крайней мере, я~верю в~это.
Аминь.
 
\clearpage
\chaps{***} 
\lettrine[lines=3, loversize=0.1]{Д}{}митрию Ивановичу совсем недавно исполнилось тридцать пять лет.
Свой день рождения он~праздновал в~компании пяти друзей, которых считал полными идиотами, и~собаки, которая досталась ему от~старушки матери и~немного от~него требовала.
Празднование вышло не~очень помпезным.
Раскрыли всего ящик плохого пива, да~одну бутылку шампанского из~супермаркета, потом до~трех ночи резались в~карты, и,~уходя, каждый гость подумал, что что-то не~так было в~этом празднике, а~может уже слишком поздно для шумных вечеринок.
Дмитрий Иванович, как с~ноткой легкого презрения произносил его имя заведующий цехом, отпраздновал свой день рождения в~пятницу утром, а~в~субботу рано утром ему уже нужно было отправляться на~работу.

Итак, он~встал в~полшестого утра в~субботу и~медленно побрел в~сторону кухни, вяло включил телевизор и~с~не~меньшей вялостью заварил себе крепкого растворимого кофе.
По~утрам он~всегда смотрел новости, но~звук предпочитал ставить на~минимальную громкость, чтобы не~слышать противного голоса ведущей, но~атмосфера пустоты и~одиночества на~кухне как бы~пряталась за~это тихое жужжание телевизора.
Обычно в~это время походкой не~менее ленивой, чем её~хозяин, на~кухне появлялась Лайма.
Дмитрий Иванович сам не~знал, почему мать так назвала свою собаку~и, если честно, не~хотел знать.
Лайму он~принял спокойно, как если бы~она подставкой для зонтиков или еще каким-нибудь предметом интерьера.
То~есть он~понимал, что Лайма существует в~его жизни только, если её~требовалось покормить или выпустить на~улицу.
Заметив собаку, он~чуть приподнял брови, словно увидел её~впервые в~жизни, что, кстати, делал всегда, когда видел Лайму.
Кто знает, может так он~приветствовал её~и~говорил спасибо за~то, что она скрашивала последние дни жизни его матери.
Дмитрий Иванович поднялся со~стула, покорно прошлепал босыми ногами по~холодному линолеуму до~двери, открыл её~и~сделал странный жест рукой вперед, как бы~говорящий:~<<Ну, иди, псина! Давай, шевелись!>>
Лайма, как и~все дни до~этого, встала ровно в~дверном проеме, принюхалась, почувствовала новый день.
Потом подняла морду на~хозяина, и~своими грустными глазами заглянула в~его, выглядевшие пустыми и~бывшие какого-то неопределенного цвета.
Так они простояли несколько мгновений, показавшихся бы~стороннему наблюдателю вечностью.
Этот ритуал повторялся каждый день.
Лайма каждый раз стояла в~дверном проеме и~жалела хозяина.
Он~был неплохим человеком, но~он~словно вовсе и~не~жил.
Лайме хотелось что-то сделать, чтобы ему помочь.
Но~что она могла? Она была всего лишь собакой.
А~люди редко слышат тех, кто ниже их~самих и~редко видят дальше собственного носа.
Потом Лайма опускала голову, разворачивалась и~топала к~выходу, свесив хвост, а~её~длинные уши подметали пол, вся ее~фигура выражала какую-то прямо человеческую боль и~скорбь по~жизни, которая могла бы~быть у~её~хозяина и~которую он~так и~не~получит.
А~Дмитрий Иванович провожал Лайму глазами, затем быстро одевался, чистил зубы, брился, если это было необходимо~и, надев повседневную маску обывателя, уходил на~работу, предварительно заперев квартиру и~положив ключ соседям под коврик.
Ему доставляла радость мысль, что у~его соседей есть ключ от~его квартиры, но~войти в~неё они не~могут, потому что не~знают об~этом.
Эта глупость заставляла уголки его губ чуть-чуть приподняться вверх и~состроить что-то наподобие улыбки.

Он~выходил на~улицу, делая вид, что идет выполнять очень важную и~ответственную работу.
До~работы ему было десять минут размеренного пешего шага, что в~молодости представляло предмет его гордости.
Но~сейчас он~бы~не~отказался от~проезда в~общественном транспорте, чтобы окунуться в~атмосферу города и~избавиться от~давящего на~душу одиночества, преследующего его уже в~течении десяти лет.
Но~вот он~уже стоял перед конфетной фабрикой, на~которой он~трудился с~того времени как его начало преследовать это ощущение одиночества и~бездарно прожитой жизни.
Как будто в~тот день, когда он~пришел на~эту фабрику, его жизнь потеряла смысл.
Конфетная фабрика располагалась в~пригороде, чтобы не~отравлять и~без того загрязненную атмосферу этого небольшого провинциального города.
Дмитрий Иванович приехал в~этот город сразу после окончания школы.
Тогда он~казался ему чуть ли~не~столицей.
ОН~поступил в~колледж, стал кем-то вроде инженера-технолога.
На~фабрике он~выполнял чуть ли~не~самую простую работу.
Он~следил за~большими машинами, которые полностью однажды вытеснили людей.
Но~машины не~вечны и~тоже когда-то выходят из~строя и~совершают ошибки.
В~его задачи входило отслеживать неисправности, а~потом исправлять их.
За~пятнадцать лет работы он~еще ни~разу ничего не~исправлял.
Машины были хорошие немецкие качественные, закупленные по~самой высокой цене.
Он~чувствовал себя сторожем.

Его рабочий день начинался в~шесть утра.
Он~переодевался в~рабочий костюм, совершал обход.
Потом садился на~маленький стульчик перед мониторами и~глупо пялился на~них до~двенадцати часов.
Двенадцать часов~---~время обеда.
За~пять минут он~доходил до~дома, запускал Лайму и~садился за~непритязательный обед, содержащий суп, макароны и~чай для него и~собачьи консервы для Лаймы.
Потом он~двадцать минут смотрел в~телевизор, не~разбирая смысл слов и~фраз, которые произносили ведущие.
А~дальше снова десять минут~---~и он~снова на~работе, теперь уже до~семи вечера.

\vspace{1cm}

Если для кого-то в~час дня уже заканчивался обед, то~для некоторых личностей в~это время только-только начиналось утро.
Вернее, для них час дня это как пять утра для школьников, учащихся в~первую смену.
Но~сегодня Алекс проснулся что-то рано, а~всему виной Мария Петровна, пришедшая раньше обычного времени.
Она была единственной, кому он~не~смел дерзить и~на~чьи слова никогда не~оговаривался.
Ей~было около шестидесяти лет, но~она все ещё была сильной и~имела огромное желание работать.
Она всегда убирала в~доме его семьи, она нянчила и~воспитывала его, знала его привычки, любимые блюда, ей~он~поверял свои тайны, от~неё единственной даже сейчас получал подзатыльники и~всегда был рад её~приходу.
Он~любил её~больше чем родную маму, которая, если честно, вспоминала о~том, что у~неё есть ребенок только тогда, когда её~подруги, сидя в~спа-салоне, спрашивали, следуя правилам приличия:<<Как там поживает твой очаровательный сынок?>>
Она сначала даже не~понимала, о~чем они говорят, и~только потом, вспомнив, что двадцать лет назад дала жизнь одному человеку~---~событие, которое она помнила только потому, что поправилась тогда на~десять килограмм, и~это её~раздражало.

Итак, Алекс проснулся в~час дня, улыбнулся Марие Петровне, получил от~неё ласковый подзатыльник и,~чуть не~запутавшись в~одеяле, прошествовал на~кухню, чтобы выпить чашечку горячего свежесваренного кофе по~всем правилам.
Он, как обычно, разблокировал телефон и~с~легкой улыбкой просмотрел новости.
Оказалось, что вчерашняя вечеринка была то, что надо.
Хорошо, что он~вовремя ушёл, иначе утро провел бы~не~так комфортабельно в~обезьяннике, как его менее удачливый друг.
Открылась кухонная дверь, вплыла Мария Петровна и, нежно посмотрев на~Алекса, предложила: <<Сашенька, солнышко, а~почему бы~тебе сегодня не~сходить на~пары в~университет? Тебе все же~следует хотя бы~иногда там появляться.
Я~знаю, что ты~скоро снова поедешь в~Англию, но~удели, пожалуйста, время своему российскому образованию.>>
Алекс улыбнулся ей, взъерошил свои волосы и, поцеловав её~в~щеку, потопал в~душ.
 
Здесь, в~России, он~вёл крайне распутную жизнь, но~там, в~Англии, он~чувствовал потребность учиться и~учился.
Там было намного интереснее, и~жизнь казалась настоящей.
А~дома он~словно погружался в~долгий сон.
Наверное, это случалось от~того, что дома он~никому не~был нужен.
Раз в~год он~на~месяц возвращался в~Россию, сдавал экзамены и~снова возвращался туда, где для него жизнь била ключом.
Жизнь здесь была для него каникулами, казавшимися порой бесконечными.
И~всё же, хотя это и~не~входило в~его правила, он~решил заглянуть в~университет в~субботу, чтобы порадовать Марию Петровну.
Как раз в~этот день в~три часа был английский.
Хорошо, можно будет поспать.
Он~гнал свой спорткар по~городу, не~обращая внимания на~знаки дорожного движения.
Всё равно он~хорошо водил машину, и~его реакция была превосходной.
 
В~шесть вечера закончилась лекция, Алекс славно поспал на~последнем ряду и~сейчас чувствовал прилив сил.
Пришло гневное СМС от~отца за~агрессивное вождение по~дороге в~университет.
Губы прорезала жестокая ухмылка.
Это его месть отцу за~пренебрежение им~в~детстве и~огромное количество любовниц, которые всегда издевались над Алексом.
Полгода назад он~сказал отцу, что тот получил сына, которого заслужил.
Отец быстро замолк.
Видимо, за~ним и~в~самом деле водились большие прегрешения.
И~снова губы Алекса расплылись в~улыбке.
На~этот раз озорной.
Он~уже предвкушал веселье.
Вечером будет грандиозное пати за~городом.
И~ему не~хватит трёх часов, чтобы как следует подготовиться.

\vspace{1cm}

Рабочий день Дмитрия Ивановича подошел к~концу.
Он~собрал бумаги, лежащие на~столе в~портфель, все остальное расставил по~местам и~приготовился ждать прихода своего коллеги.
Тот, как обычно, опоздал на~пятнадцать минут и~был слегка пьян.
Дмитрий Иванович пожелал ему удачи и~удрученно пошел домой.
Зайдя в~квартиру, он~разулся, присел на~табурет и~застыл, погруженный в~свои мысли.
Так он~сидел ровно до~того момента, когда к~нему подошла Лайма и~уткнула свой черный нос в~его колени.
Он~перевел свой невидящий взор на~собаку и~вновь застыл.
Это был их~ежевечерний ритуал, повторявшийся со~дня смерти мамы.
Потом человек поднялся и~прошел на~кухню, где, сварив себе кофе и дав консервов собаке, включил телевизор.

\vspace{1cm}

В~девять вечера автомобиль Алекса, въехал в~ворота милого особняка, располагавшегося за~городом, который был выбран местом действия сегодняшней вечеринки.
Алекс быстро вбежал по~лестнице и~встретил друга, обменявшись с~ним рукопожатиями, он~поднялся на~третий этаж, где находилось то, за~чем собственно Алекс и~приехал.
А~именно библиотека, с~редкими изданиями, которую собирал отец Кира (друга Алекса).
Кир мало интересовался книгами и~прочей лабудой, как сам любил говорить.
Ему намного веселее и~интереснее было в~компании девушек, модельной внешности и~с~таким же, как у~моделей, размером мозга.
Но~Алекс приехал сюда, чтобы встретиться с~немецкой философией в~оригинале, а~не~потусить.
Кир об~этом знал, но~было решено еще давным давно поддерживать репутацию взбалмошного идиота, занятого только развлечениями, которую Алекс создавал годы, чтобы позлить отца.

\vspace{1cm}

В~девять вечера Дмитрий Иванович выключил телевизор и~погасил свет.
Для всего мира он~отошел ко~сну.
Но~на~самом деле он~прошествовал в~комнату матери, задернул плотные шторы, сдвинул книжный шкаф к~окну так, чтобы с~улицы не~было, что у~него включен свет.
По~ночам он~занимался святотатсвом, потому что включал ноутбук, выходил в~интернет и~слушал лекции ведущих физиков мира, как на~русском, так и~на~английском.
Физика была всегда его страстью, но~реализовать себя в~ней не~получилось, поэтому оставалось лишь быть в~курсе последних достижений любимой науки.

\vspace{1cm}

Гости начали приходить, включилась музыка.
Это означало, что книги нужно брать и~уходить на~крышу.
Майская ночь была чудесной и~удивительно свежей.
Самое время, чтобы заниматься Марксом и~Ницше.
Внезапно Алекс услышал чьи-то шаги.
Легкое шуршание и~удивленный вскрик «Ой!».
Алекс раздраженно повернулся к~тому, кто издавал звук, в~надежде заставить его уйти и~самому погрузиться в~книги.
Это оказалась девушка.
Не~модель.
Маленькая.
С~пухлыми щеками и~серьезными глазами.
Она, сделав невозмутимое выражение лица, подошла к~краю, где сидел Алекс, и~с~самым независим видом уселась там.
Девушка подняла глаза к~небу и~посмотрела на~звезды.
Так они и~сидели.
Она, погруженная в~мир звезд, что всегда давали людям вдохновение и~надежду.
И~он, погруженный в~мир книг, надеявшийся найти вдохновение в~нем, в~то~время как был близок к~источнику истинного вдохновения и~так же~далек от~него.

\vspace{1cm}

Эта девушка~---~я.
Я~жила в~маленьком провинциальном городке в~соседней от~Дмитрия Ивановича квартире.
Я~дружила с~его собакой Лаймой, и~мы~вместе разговаривали о~её~хозяине.
Это я~любила танцы, а~потом училась чувствовать жизнь заново.
Это меня встретил человек в~автобусе, и~я~плакала от~того, что в~тот день окончательно смогла вернуть себе чувства и~не~знала как теперь жить и~какая же~цель должна быть теперь в~моей жизни.
Старая профессор, что бьет ногами лед в~парке~---~мой преподаватель физики в~университете в~Москве, и~она по-настоящему чудесный учитель, которого все мы~очень любим.
А~Алекса я~встретила вчера ночью.
Мы~с~ним поговорили о~многом.
Он~несчастный человек, который был обижен родителями всю свою жизнь.
Он~прекрасно понимает, что бороться с~этим нужно, но~не~знает как.
Я~решила попробовать воссоздать свою нить жизни, но~она будет состоять из~людей, встреча с~которыми не~то, чтобы случайна, но~на~их~месте могли бы~быть и~другие.

\vspace{1cm}

\textit{
P.~S.~Я~писала эти короткие зарисовки жизни, чтобы понять, что значит для меня душевная красота.
Это сугубо личный взгляд на~эти понятия.
Мне хотелось понять, что кроме обычных благородства, доброты и~сострадания должно быть в~человеке, чтобы я~смогла заинтересоваться им~и~сказать, что для меня он~внутренне красив.
Начнем с~умения ценить красоту и~настоящее искусство и~находить их~в~повседневности.
Потом будут идти рыцарские устремления и~проницательность.
Дальше ворчливость, капля цинизма пополам со~здравым смыслом, мудрость вкупе с~ребячливостью.
Но~куда же~без грусти и~одиночества, преданности любимому делу и~желанием быть в~курсе всего связанного с~ним, не~имея надежды, когда ли~полно его познать.
А~также бесконечная, затмевающая все любовь к~книгам и~звездам~---~самым преданным источникам вдохновения и~всего, что есть в~людях наилучшего.
}